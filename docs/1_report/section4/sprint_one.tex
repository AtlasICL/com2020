\subsection{Explanation}
Each subsection is an aspect of the project (either the documentation, python or java application). Each subsubsection is then an epic within that area. Each paragraph represents one story for that epic, referencing a story in the project backlog. Each story is assigned a value associated with how much work is required to complete the story for this sprint, the risks associated with the story, the technical changes required to complete it, what it depends on, how complex it is to achieve, and what value it adds to the project, assuming all work to be done in the last sprint was done. Each story is also assigned a priority of either Must, Should, could and wont. Lastly an acceptance criteria is provided for each story, stating what tasks must be completed to achieve the story. 

\subsection{Documentation}

\subsubsection{Report}

\paragraph{Stakeholders can View Executive Summary}
\textbf{Value:} 2 \quad \textbf{Priority:} Must\\
\textbf{Acceptance Criteria}
\begin{itemize} 
    \item The summary appears at the start of the report.
    \item The summary clearly states what is delivered in the prototype.
    \item The summary clearly states why prototype features exist in the prototype. 
    \item The summary clearly states the most important risks of the prototype.
    \item The summary clearly states the mitigations for those risks.
\end{itemize}

\paragraph{Stakeholders can View Problem Framing}
\textbf{Value:} 3\quad \textbf{Priority:} Must\\
\textbf{Acceptance Criteria}
\begin{itemize} 
    \item The problem framing clearly explains the problem.
    \item The problem framing clearly states why the problem is important.
    \item The problem framing clearly states what the goal of the project is.
    \item The problem framing must state who is affected by the problem. 
\end{itemize}

\paragraph{Stakeholders can View Definition of Done}
\textbf{Value:} 1\quad \textbf{Priority:} Must\\
\textbf{Acceptance Criteria}
\begin{itemize} 
    \item The definition of done must clearly explain when a documentation story is complete.
    \item The definition of done must clearly explain when a program story is complete.
\end{itemize}

\paragraph{Stakeholders can View Project Backlog}
\textbf{Value:} 3\quad \textbf{Priority:} Must\\
\textbf{Acceptance Criteria}
\begin{itemize} 
    \item The project backlog must clearly state which epics are required to complete the project.
    \item The project backlog must clearly state which user stories are associated with each epic.
\end{itemize}

\paragraph{Stakeholders can View Sprint One Prioritised Requirements}
\textbf{Value:} 3\quad \textbf{Priority:} Must\\
\textbf{Acceptance Criteria}
\begin{itemize} 
  \item Each epic in the project backlog must be stated in the requirements.
  \item Each story within each epic must be stated in the requirements.
  \item Each story must have a value associated with it, based on how much work it is to complete the story, the significance of the risks associated with it, the technical changes required to complete it, how much depends on it, how complex it is to complete, and the value it's completion adds to the project.
  \item Each story must have an acceptance criteria associated with it, which when all criteria are met, the story is considered complete for sprint one. Criteria may change between sprints as sprint requirements change. 
  \item Each story must have a priority associated with it for that sprint. The priority must either be Must, Should, Could or Wont. 
  \item Each story must be able to link to a prototype feature. 
\end{itemize}


\paragraph{Stakeholders can View Architecture Schema}
\paragraph{Architecture Schema}
\textbf{Value:} 3\quad \textbf{Priority:} Should\\
\textbf{Acceptance Criteria}
\begin{itemize} 
  \item Each class category of the Java application must have its purpose and use case explained.
  \item Each class category of the Python application must have its purpose and use case explained. 
  \item Classes that share an identical structure with minor variation need only for their category to be explained.
  \item All structural design choices must be justified and explained.
  \item The limitations of all specified design choices must be stated, and why they're acceptable must be explained.
\end{itemize}

\paragraph{Stakeholders can View Data Flow Schema}
\textbf{Value:} 2\quad \textbf{Priority:} Should\\
\textbf{Acceptance Criteria}
\begin{itemize} 
  \item Each component that accesses data should be explained, describing what data it accesses, what it does with it, and why it needs to access this data.
  \item Each data store should be explained, stating what data it stores and in what format. 
\end{itemize}

\paragraph{Stakeholders can View Telemetry Schema}
\textbf{Value:} 3\quad \textbf{Priority:} Must\\
\textbf{Acceptance Criteria}
\begin{itemize} 
  \item Each telemetry event must be described by the schema.
  \item Each telemetry event must have all it's fields described by the schema.
  \item Each field must have its domain described by the schema. For enumerated fields all valid values must be described.
\end{itemize}

\paragraph{Stakeholders can View Game Design Documentation}
\textbf{Value:} 3\quad \textbf{Priority:} Could\\
\textbf{Acceptance Criteria}
\begin{itemize} 
  \item The documentation must outline the gameplay loop.
  \item The documentation must explain all parts of the gameplay loop.
  \item The documentation must define all attack abilities (both magic and mundane) along with their cost to purchase, magic cost, damage type, and base damage.
  \item The documentation must define all passive abilities, what they do, and their cost.
  \item The documentation must define all damage types and their use cases.
  \item The documentation must outline a structure for the game's progression.
  \item The documentation must define all enemies within the game.

\end{itemize}

\paragraph{Stakeholder can View Evaluation}
\textbf{Value:} 3\quad \textbf{Priority:} Must\\
\textbf{Acceptance Criteria}
\begin{itemize} 
  \item The evaluation must define 5 measures of success.
  \item For each measure, evidence must be provided to show system effectiveness.
  \item The limitations and biases of the evaluation must also be explored and explained.
\end{itemize}

\paragraph{Stakeholders can View Sprint Two Prioritised Requirements}
\textbf{Value:} 3\quad \textbf{Priority:} Should\\
\textbf{Acceptance Criteria}
\begin{itemize}
  \item Each epic in the project backlog must be stated in the requirements.
  \item Each story within each epic must be stated in the requirements.
  \item Each story should have a value associated with it, based on how much work it is to complete the story, the significance of the risks associated with it, the technical changes required to complete it, how much depends on it, how complex it is to complete, and the value it's completion adds to the project.
  \item Each story must have an acceptance criteria associated with it, which when all criteria are met the story is considered complete for sprint two. Criteria may change between sprints as sprint requirements change. 
  \item Each story must have a priority associated with it for that sprint. The priority must either be Must, Should, Could or Wont. 
  \item Each story must be able to link to a final product feature and test. 
\end{itemize} 

\paragraph{Stakeholders can View Management}
\textbf{Value:} 2\quad \textbf{Priority:} Should\\
\textbf{Acceptance Criteria}
\begin{itemize}
  \item The management section must outline what roles each member had.
  \item It must outline what tasks members owned, and what tasks they contributed to.
  \item It must outline in what areas the group worked well together.
  \item It must outline what challenges the group faced working together.
  \item It must outline how those challenges were solved. 
\end{itemize}


\subsubsection{Process Information}

\paragraph{Stakeholders can View Meeting Minutes}
\textbf{Value:} 3\quad \textbf{Priority:} Must\\
\textbf{Acceptance Criteria}
\begin{itemize}
    \item The meeting minutes must state who attended each meeting.
    \item They must state what work was completed by each group member who attended.
    \item They must state what topics were discussed in each meeting.
    \item They must state what work is to be done in the next weekly scrum, and by who. 
\end{itemize}

\paragraph{Stakeholder can view Risk Register}
\textbf{Value:} 5\quad \textbf{Priority:} Must\\
\textbf{Acceptance Criteria}
\begin{itemize}
    \item Each risk must be assigned a unique identifier.
    \item Each risk must be given a clear description.
    \item The consequences of each risk occurring must be clearly stated.
    \item The ordinal impact, from 1 to 5, must be stated for each risk.
    \item The ordinal likelihood, from 1 to 5, must be stated for each risk.
    \item The ordinal value, from 1 to 25, equalling the product of the impact and likelihood of the risk, must be stated.
    \item The treatment for each risk must be stated, including treatment method (accept, transfer, mitigate, avoid) and details of the treatment. 
    \item An owner for each risk must be assigned and recorded.
\end{itemize}

\subsubsection{Ethical and Legal Considerations}

\paragraph{Stakeholders can View the Privacy and Data Protection Analysis}
\textbf{Value:} 1\quad \textbf{Priority:} Must\\
\textbf{Acceptance Criteria}
\begin{itemize}
  \item Must discuss what data is stored about users.
  \item Must discuss how the data storage complies with law.
  \item Must discuss how collected data is pseudonymised.
  \item Must discuss how access to this data is limited to only those it's relevant to.
\end{itemize}

\paragraph{Stakeholders can View the Consent and Disclosure Analysis}
\textbf{Value:} 1\quad \textbf{Priority:} Must\\
\textbf{Acceptance Criteria}
\begin{itemize}
  \item Must discuss what the data is used for.
  \item Must discuss how consent is handled.
  \item Must discuss how this consent and disclosure policy complies with law.
  \item Must discuss how this policy affects the user experience. 
\end{itemize}

\paragraph{Stakeholders can View the Accessibility Analysis}
\textbf{Value:} 1\quad \textbf{Priority:} Should\\
\textbf{Acceptance Criteria}
\begin{itemize}
  \item Must discuss the importance of accessability.
  \item Must discuss what systems are in place to support the accessability of the software.
  \item Must discuss what areas the system can improve upon for sprint 2.
\end{itemize}

\paragraph{Stakeholders can View the Intellectual Property and Licensing Implications}
\textbf{Value:} 1\quad \textbf{Priority:} Must\\
\textbf{Acceptance Criteria}
\begin{itemize}
  \item Must discuss the intellectual property rights of the developers.
  \item Must discuss the intellectual property rights of the product owner.
  \item Must discuss the licensing considerations for the software and data the system depends on.
  \item Must discuss the intellectual property rights of sources the systems draws inspiration from or is similar to.
\end{itemize}

\subsubsection{Project License}

\paragraph{Stakeholders can View the Project License}
\textbf{Value:} 1\quad \textbf{Priority:} Must\\
\textbf{Acceptance Criteria}
\begin{itemize}
  \item A license for the project must be determined.
  \item The license must comply with the licensing requirements of all dependencies. 
  \item The reasons for the choice of license must be explained. 
\end{itemize} 

\subsubsection{Software and Data Inventory}

\paragraph{Maintainers can View the Software Inventory}
\textbf{Value:} 2\quad \textbf{Priority:} Should\\
\textbf{Acceptance Criteria}
\begin{itemize}
  \item An inventory of all software components the system directly depends on must be produced.
  \item This must contain the license of each dependency.
  \item This must contain the cost model of each dependency.
  \item This must contain the provenance of each dependency.
  \item This must contain the version of each dependency the system uses.
\end{itemize}

\paragraph{Maintainers can View the Data Inventory}
\textbf{Value:} 1\quad \textbf{Priority:} Should\\
\textbf{Acceptance Criteria}
\begin{itemize}
  \item An inventory of all data the system directly depends on or uses must be produced.
  \item This must contain the license of each dependency.
  \item This must contain the cost model of each dependency.
  \item This must contain the provenance of each dependency.
  \item This must contain the version of each dependency the system uses.
\end{itemize}

\subsubsection{Deployment and Operations Guide}

\paragraph{Client can View the Java Application Deployment Instructions}
\textbf{Value:} 1\quad \textbf{Priority:} Must\\
\textbf{Acceptance Criteria}
\begin{itemize}
  \item The guide must contain clear instructions for how to set up and operate the Java application.
  \item There may be instructions for different systems, but there should be a way to run the application on any desktop system that runs Windows, MacOS or Linux.
\end{itemize}

\paragraph{Client can View the Python Application Deployment Instructions}
\textbf{Value:} 1\quad \textbf{Priority:} Must\\
\textbf{Acceptance Criteria}
\begin{itemize}
  \item The guide must contain clear instructions for how to set up and operate the Python application.
  \item There may be instructions for different systems, but there should be a way to run the application on any desktop system that runs Windows, MacOS or Linux.
\end{itemize}

\paragraph{Maintainer can View the Automated Test Running Instructions}
\textbf{Value:} 1\quad \textbf{Priority:} Must\\
\textbf{Acceptance Criteria}
\begin{itemize}
  \item There should be clear instructions on how to operate the test suite on the source code.
  \item The version of the testing framework should be provided.
  \item The running instructions should allow the running of the test on any desktop system that runs either Windows, MacOS or Linux.
\end{itemize}

\subsubsection{Test Evidence}

\paragraph{Stakeholder can View the Automated Test Evidence}
\textbf{Value:} 2\quad \textbf{Priority:} Must\\
\textbf{Acceptance Criteria}
\begin{itemize}
  \item There must be 5 automated tests.
  \item Some of them must unit tests.
  \item Others must be integration tests.
\end{itemize}

\paragraph{Stakeholder can View the End to End Test Evidence}
\textbf{Value:} 1\quad \textbf{Priority:} Must\\
\textbf{Acceptance Criteria}
\begin{itemize}
  \item There must be one end to end test.
  \item Evidence that the test is successful should be given.
  \item The happy path and failure cases for the test must be described.
  \item The expected results of the test should be described.
\end{itemize}

\subsubsection{Presentation}

\paragraph{Stakeholder can View an Explanation of the Problem}
\textbf{Value:} 2\quad \textbf{Priority:} Must\\
\textbf{Acceptance Criteria}
\begin{itemize}
  \item Must outline what the problem is.
  \item Must outline why the problem is relevant.
  \item Must outlie the scope of the problem.
  \item Must outline the scope of the solution.
  \item Must outline what assumptions were made about the problem.
  \item Must outline who the intended users of the solution are. 
\end{itemize}

\paragraph{Stakeholder can View the Design Approach}
\textbf{Value:} 2\quad \textbf{Priority:} Must\\
\textbf{Acceptance Criteria}
\begin{itemize}
  \item Must outline the architecture of the solution.
  \item Must outline measures used to evaluate the system.
\end{itemize}

\paragraph{Stakeholder can View Details of the Implementation}
\textbf{Value:} 2\quad \textbf{Priority:} Must\\
\textbf{Acceptance Criteria}
\begin{itemize}
  \item Must outline the key components of the solution.
  \item Must outline the key features of the implementation.
  \item Must outline why key design decisions were made.
\end{itemize}

\paragraph{Stakeholder can View the Evaluation}
\textbf{Value:} 2\quad \textbf{Priority:} Must\\
\textbf{Acceptance Criteria}
\begin{itemize}
  \item Must outline what issues were faced during implementation.
  \item Must outline how well the system performs against 5 measures, with clear evidence.
\end{itemize}

\paragraph{Stakeholder can View a Live Demonstration}
\textbf{Value:} 3\quad \textbf{Priority:} Must\\
\textbf{Acceptance Criteria}
\begin{itemize}
  \item Must take around 4 minutes to complete.
  \item Must demonstrate a run of the game.
  \item Must demonstrate the telemetry views.
  \item Must demonstrate the telemetry suggestions.
  \item Must demonstrate the simulation.
  \item Must demonstrate the adjustment of a design parameter.
  \item Must be rehearsed before the live demo, to ensure it works correctly.
  \item Must have a backup demonstration, should it go wrong.
\end{itemize}

\paragraph{Stakeholder can View the Limitations and Next Steps}
\textbf{Value:} 2\quad \textbf{Priority:} Must\\
\textbf{Acceptance Criteria}
\begin{itemize}
  \item Must outline the areas in which the prototype is weakest.
  \item Must briefly outline what is to be done in the next sprint.
\end{itemize}

\paragraph{Stakeholder can View the Ethics and Telemetry Disclosure}
\textbf{Value:} 0\quad \textbf{Priority:} Wont\\
\textbf{Acceptance Criteria} N/A

\paragraph{Stakeholders can View Management Information}
\textbf{Value:} 1\quad \textbf{Priority:} Must\\
\textbf{Acceptance Criteria}
\begin{itemize}
  \item Must outline the role of each member.
  \item Must outline what each member contributed to the project.
\end{itemize}

\paragraph{Stakeholders can View an Brief of the Scrum Board}
\textbf{Value:} 0\quad \textbf{Priority:} Wont\\
\textbf{Acceptance Criteria} N/A

\subsubsection{Maintenance and Troubleshooting Guide}

\paragraph{Maintainers can View Maintenance Instructions}
\textbf{Value:} 0\quad \textbf{Priority:} Wont\\
\textbf{Acceptance Criteria} N/A

\paragraph{Users can View Troubleshooting Instructions}
\textbf{Value:} 0\quad \textbf{Priority:} Wont\\
\textbf{Acceptance Criteria} N/A

\paragraph{Maintainers can View Extension Instructions}
\textbf{Value:} 0\quad \textbf{Priority:} Wont\\
\textbf{Acceptance Criteria} N/A

\subsubsection{Data Management Guide}

\paragraph{Stakeholders can View What Data is Stored}
\textbf{Value:} 0\quad \textbf{Priority:} Wont\\
\textbf{Acceptance Criteria} N/A

\paragraph{Maintainers can View the Data Format}
\textbf{Value:} 0\quad \textbf{Priority:} Wont\\
\textbf{Acceptance Criteria} N/A

\subsubsection{Scrum Board}

\paragraph{Stakeholders can View a Snapshot of the Scrum Board}
\textbf{Value:} 3\quad \textbf{Priority:} Must\\
\textbf{Acceptance Criteria} 
\begin{itemize}
  \item There must be a card for each story in the project backlog defined in this report.
  \item There must be a card for each group of acceptance criteria (task) that varies between sprints.
  \item There must be epics created on the board for each epic defined in the project backlog in this report.
  \item There must be a backlog section for tasks that have yet to have been started.
  \item There must be an in progress section for tasks being worked on. Each task in this section must be assigned to at least one group member.
  \item There must be a done section for tasks that are complete. Each task in this section must be assigned to at least one group member.
\end{itemize}

\paragraph{Stakeholders can View a Snapshot of the Scrum Timeline}
\textbf{Value:} 1\quad \textbf{Priority:} Could\\
\textbf{Acceptance Criteria} 
\begin{itemize}
  \item Must show when each epic was started and its current progress.
  \item Must show the progress on different cards (stories). 
\end{itemize}

\subsection{Python Application}

\subsubsection{Authentication and Access Control}

\paragraph{Users can Log In}
\textbf{Value:} 5\quad \textbf{Priority:} Must\\
\textbf{Acceptance Criteria} 
\begin{itemize}
  \item Users must be required to log in before accessing the python application.
  \item Users with an account must be able to log into that account by entering the correct details.
  \item Users without an account must be able to create one.
\end{itemize}

\paragraph{Users can Reset their Login}
\textbf{Value:} 1\quad \textbf{Priority:} Could\\
\textbf{Acceptance Criteria}
\begin{itemize}
  \item Users must be able to reset their login credentials.
  \item This reset system must require them to have some other security factor such as email.
\end{itemize}

\paragraph{Users can Perform Actions Permitted by Their Role}
\textbf{Value:} 2\quad \textbf{Priority:} Must\\
\textbf{Acceptance Criteria}
\begin{itemize}
    \item A user with the player role must not be able to access the application.
    \item A user with the designer or developer role must have full access to the application.
\end{itemize}

\paragraph{Users can view their Profile}
\textbf{Value:} 0\quad \textbf{Priority:} Wont\\
\textbf{Acceptance Criteria} N/A

\subsubsection{Telemetry Views}

\paragraph{Designers can View Funnel}
\textbf{Value:} 3\quad \textbf{Priority:} Must\\
\textbf{Acceptance Criteria}
\begin{itemize}
  \item The system must calculate a funnel view showing user character drop off from the telemetry data as the game progresses.
  \item This must then be displayed to the user. 
  \item It must be able to be switched from viewing actual user data, to simulation generated telemetry data.
\end{itemize}

\paragraph{Designers can View Difficulty Spikes}
\textbf{Value:} 3\quad \textbf{Priority:} Must\\
\textbf{Acceptance Criteria}
\begin{itemize}
  \item The system must calculate a difficulty spikes view showing failure rate for different stages from the telemetry data.  
  \item This must then be displayed to the user. 
  \item It must be able to be switched from viewing actual user data, to simulation generated telemetry data.
\end{itemize}

\paragraph{Designers can View Progress Curves}
\textbf{Value:} 3 \quad \textbf{Priority:} Could\\
\textbf{Acceptance Criteria} 
\begin{itemize}
  \item The system must calculate how many coins user's character's accumulated on each stage from the telemetry data.
  \item This must then be displayed to the user. 
  \item It must be able to be switched from viewing actual user data, to simulation generated telemetry data.
\end{itemize}

\paragraph{Designers can View Fairness Indicators}
\paragraph{Fairness Indicators}
\textbf{Value:} 0 \quad \textbf{Priority:} Wont\\
\textbf{Acceptance Criteria} N/A

\paragraph{Designers can View Comparison Mode}
\textbf{Value:} 3 \quad \textbf{Priority:} Could\\
\textbf{Acceptance Criteria} 
\begin{itemize}
  \item The system must calculate how many coins user's character's accumulated on each stage along with how much health they finish each stage with, for each difficulty.
  \item This must then be displayed to the user. 
  \item It must be able to be switched from viewing actual user data, to simulation generated telemetry data.
\end{itemize}

\paragraph{Designers can View Rule Based Design Suggestions}
\textbf{Value:} 2 \quad \textbf{Priority:} Should\\
\textbf{Acceptance Criteria} 
\begin{itemize}
  \item The system should read the telemetry data and evaluate it against a suggestion criteria.
  \item If this criteria is met, the system should suggest a concrete change to the design parameter to improve the balance of the game. 
  \item This suggestions should be displayed to the user.
\end{itemize}

\paragraph{Designers can View Decision Log}
\textbf{Value:} 0 \quad \textbf{Priority:} Wont\\
\textbf{Acceptance Criteria} N/A

\subsubsection{Telemetry Events}

\paragraph{Application Handles Start Session}
\textbf{Value:} 1 \quad \textbf{Priority:} Must\\
\textbf{Acceptance Criteria} 
\begin{itemize}
    \item The application must be able to read start session events that follow the telemetry schema.
    \item The application must be able to validate those events.
    \item The application must be provide clear errors to the user when failing to validate them.
\end{itemize}

\paragraph{Application Handles Normal Encounter Fail}
\textbf{Value:} 1 \quad \textbf{Priority:} Must\\
\textbf{Acceptance Criteria} 
\begin{itemize}
    \item The application must be able to read normal encounter fail events that follow the telemetry schema.
    \item The application must be able to validate those events.
    \item The application must be provide clear errors to the user when failing to validate them.
\end{itemize}

\paragraph{Application Handles Normal Encounter Complete}
\textbf{Value:} 1 \quad \textbf{Priority:} Must\\
\textbf{Acceptance Criteria} 
\begin{itemize}
    \item The application must be able to read normal encounter complete events that follow the telemetry schema.
    \item The application must be able to validate those events.
    \item The application must be provide clear errors to the user when failing to validate them.
\end{itemize}

\paragraph{Application Handles Boss Encounter Start}
\textbf{Value:} 1 \quad \textbf{Priority:} Could\\
\textbf{Acceptance Criteria} 
\begin{itemize}
    \item The application must be able to read boss encounter start events that follow the telemetry schema.
    \item The application must be able to validate those events.
    \item The application must be provide clear errors to the user when failing to validate them.
\end{itemize}

\paragraph{Application Handles Boss Encounter Fail}
\textbf{Value:} 1 \quad \textbf{Priority:} Could\\
\textbf{Acceptance Criteria} 
\begin{itemize}
    \item The application must be able to read boss encounter fail events that follow the telemetry schema.
    \item The application must be able to validate those events.
    \item The application must be provide clear errors to the user when failing to validate them.
\end{itemize}

\paragraph{Application Handles Boss Encounter Complete}
\textbf{Value:} 1 \quad \textbf{Priority:} Could\\
\textbf{Acceptance Criteria} 
\begin{itemize}
    \item The application must be able to read boss encounter complete events that follow the telemetry schema.
    \item The application must be able to validate those events.
    \item The application must be provide clear errors to the user when failing to validate them.
\end{itemize}

\paragraph{Application Handles Gain Coin}
\textbf{Value:} 1 \quad \textbf{Priority:} Could\\
\textbf{Acceptance Criteria} 
\begin{itemize}
    \item The application must be able to read gain coin events that follow the telemetry schema.
    \item The application must be able to validate those events.
    \item The application must be provide clear errors to the user when failing to validate them.
\end{itemize}

\paragraph{Application Handles Buy Upgrade}
\textbf{Value:} 1 \quad \textbf{Priority:} Could\\
\textbf{Acceptance Criteria} 
\begin{itemize}
    \item The application must be able to read buy upgrade events that follow the telemetry schema.
    \item The application must be able to validate those events.
    \item The application must be provide clear errors to the user when failing to validate them.
\end{itemize}

\paragraph{Application Handles End Session}
\textbf{Value:} 1 \quad \textbf{Priority:} Must\\
\textbf{Acceptance Criteria} 
\begin{itemize}
    \item The application must be able to read end session events that follow the telemetry schema.
    \item The application must be able to validate those events.
    \item The application must be provide clear errors to the user when failing to validate them.
\end{itemize}

\paragraph{Application Handles Settings Change}
\textbf{Value:} 1 \quad \textbf{Priority:} Must\\
\textbf{Acceptance Criteria} 
\begin{itemize}
    \item The application must be able to read settings change events that follow the telemetry schema.
    \item The application must be able to validate those events.
    \item The application must be provide clear errors to the user when failing to validate them.
\end{itemize}

\paragraph{Application Handles Kill Enemy}
\textbf{Value:} 1 \quad \textbf{Priority:} Could\\
\textbf{Acceptance Criteria} 
\begin{itemize}
    \item The application must be able to read kill enemy events that follow the telemetry schema.
    \item The application must be able to validate those events.
    \item The application must be provide clear errors to the user when failing to validate them.
\end{itemize}

\subsubsection{Seeded Dataset}

\paragraph{Developers can Access a Telemetry Dataset}
\textbf{Value:} 0 \quad \textbf{Priority:} Wont\\
\textbf{Acceptance Criteria} N/A

\paragraph{Developers can Access a Decision Dataset}
\textbf{Value:} 0 \quad \textbf{Priority:} Wont\\
\textbf{Acceptance Criteria} N/A

\subsection{Java Application}

\subsubsection{Autheentication and Access Control}

\paragraph{User can Log In}
\textbf{Value:} 5\quad \textbf{Priority:} Must\\
\textbf{Acceptance Criteria} 
\begin{itemize}
  \item Users must be required to log in before accessing the Java application.
  \item Users with an account must be able to log into that account by entering the correct details.
  \item Users without an account must be able to create one.
\end{itemize}

\paragraph{Users can Reset their Login}
\textbf{Value:} 1\quad \textbf{Priority:} Could\\
\textbf{Acceptance Criteria}
\begin{itemize}
  \item Users must be able to reset their login credentials.
  \item This reset system must require them to have some other security factor such as email.
\end{itemize}

\paragraph{Users can Perform Actions Permitted by Their Role}
\textbf{Value:} 2\quad \textbf{Priority:} Must\\
\textbf{Acceptance Criteria}
\begin{itemize}
    \item A user with the player role must be able to toggle telemetry, view the telemetry disclosure, and play the game.
    \item A user with the designer or developer role must be able to view the values of design parameters and execute simulated runs.
\end{itemize}

\subsubsection{Settings}

\paragraph{Users can View the Telemetry Disclosure}
\textbf{Value:} 1\quad \textbf{Priority:} Must\\
\textbf{Acceptance Criteria}
\begin{itemize}
    \item A user in the main menu must be able to view the telemetry disclosure.
    \item A user in settings must be able to view the telemetry disclosure.
    \item The telemetry disclosure must contain what data is collected.
    \item The telemetry disclosure must explain why that data is collected.
\end{itemize}

\paragraph{Users can Toggle Telemetry}
\textbf{Value:} 1\quad \textbf{Priority:} Must\\
\textbf{Acceptance Criteria}
\begin{itemize}
    \item A user in settings must be able to see if they have telemetry enabled or not.
    \item A user in settings must be able to toggle whether they have telemetry enabled or not.
    \item While disabled telemetry events should not be recorded.
    \item It must by default be on
    \item The setting must persist beyond the lifetime of the game.
\end{itemize}

\paragraph{Designers can View and Change Starting Lives}
\textbf{Value:} 2\quad \textbf{Priority:} Must\\
\textbf{Acceptance Criteria}
\begin{itemize}
    \item A designer in settings must be able to view the starting lives for each difficulty.
    \item A designer in settings must be able to set the value of starting lives for each difficulty.
    \item The value for starting lives must persist beyond the lifetime of the game.
\end{itemize}

\paragraph{Designers can View and Change Enemy Health Multiplier}
\textbf{Value:} 0\quad \textbf{Priority:} Wont\\
\textbf{Acceptance Criteria} N/A

\paragraph{Designers can View and Change Player Maximum Health}
\textbf{Value:} 0\quad \textbf{Priority:} Wont\\
\textbf{Acceptance Criteria} N/A

\paragraph{Designers can View and Change Upgrade Price Multiplier}
\textbf{Value:} 0\quad \textbf{Priority:} Wont\\
\textbf{Acceptance Criteria} N/A

\paragraph{Designers can View and Change Enemy Damage Multiplier}
\textbf{Value:} 0\quad \textbf{Priority:} Wont\\
\textbf{Acceptance Criteria} N/A

\paragraph{Designers can View and Change Player Maximum Magic}
\textbf{Value:} 0\quad \textbf{Priority:} Wont\\
\textbf{Acceptance Criteria} N/A

\paragraph{Designers can View and Change Magic Regeneration Rate}
\textbf{Value:} 0\quad \textbf{Priority:} Wont\\
\textbf{Acceptance Criteria} N/A

\paragraph{Designers can View and Change Number of Items in the Shop}
\textbf{Value:} 0\quad \textbf{Priority:} Wont\\
\textbf{Acceptance Criteria} N/A

\paragraph{Designers can Execute a Simulated Run}
\textbf{Value:} 2\quad \textbf{Priority:} Should\\
\textbf{Acceptance Criteria} 
\begin{itemize}
  \item A designer must be able to execute a simulated run.
  \item The run must use a player simulation instead of a user player, and enemies must behave as normal.
  \item The player agent must be able to perform any action a user player could.
  \item The run should continue until the agent beats the game or dies.
\end{itemize}

\paragraph{Developers can Assign Roles}
\textbf{Value:} 5\quad \textbf{Priority:} Could\\
\textbf{Acceptance Criteria} 
\begin{itemize}
  \item An authenticated developer can enter the username of another user and assign a role to them.
  \item The first user must be a developer.
  \item There must always be one developer account.
\end{itemize}

\subsubsection{Game Runs}

\paragraph{User can Chose Difficulty}
\textbf{Value:} 3\quad \textbf{Priority:} Must\\
\textbf{Acceptance Criteria} 
\begin{itemize}
  \item At the start of a game run the player can select the difficulty for the run.
  \item The difficulty options are Easy, Medium and Hard.
  \item The game must be harder on each subsequent difficulty options (easy is the easiest, hard is the hardest).
  \item The game must vary how hard a difficulty is by varying the values of design parameters. 
\end{itemize}

\paragraph{Player can Use Mundane Attack Abilities}
\textbf{Value:} 8\quad \textbf{Priority:} Must\\
\textbf{Acceptance Criteria} 
\begin{itemize}
  \item The player must always be able to use the default attack ability on their turn while in an encounter. 
  \item The player must be able to use any mundane attack abilities they've purchased in the shop on their turn in an encounter
  \item These abilities must deal the damage type specified in the game design documentation.
  \item These abilities must deal an amount of damage based on the game design documentation, to the enemy. 
  \item The enemy's health must be reduced by this damage amount.
  \item Any purchasable mundane attack abilities must be purchasable in the shop between encounters.
  \item They must not be purchasable if the player does not have enough coins.
  \item They must reduce the amount of coins the player has by their cost when purchased.
  \item They must only be purchased once per run.
\end{itemize}

\paragraph{Player can Chose Which Abilities to Purchase}
\textbf{Value:} 1\quad \textbf{Priority:} Should\\
\textbf{Acceptance Criteria} 
\begin{itemize}
  \item Multiple abilities must appear in the shop between encounters.
  \item Players must be able to chose which ability from the shop they wish to buy.
  \item Player must be able to purchase multiple abilities from the same shop.
\end{itemize}

\paragraph{Player Can Use Purchased Magic Attack Abilities}
\textbf{Value:} 8\quad \textbf{Priority:} Could\\
\textbf{Acceptance Criteria} 
\begin{itemize} 
  \item Players must have a finite magic reserve, limited by their maximum magic.
  \item While in an encounter, if the player has sufficient magic they must be able to use a magic ability on their turn if they've purchased it.
  \item If the player doesn't have enough magic to fuel a magic attack ability they mustn't be able to use it.
  \item When a magic attack ability is used it must reduce their magic by its cost as defined in the game design documentation.
  \item These abilities must deal the damage type specified in the game design documentation.
  \item These abilities must deal an amount of damage based on the game design documentation, to the enemy. 
  \item The enemy's health must be reduced by this damage amount.
  \item Any purchasable magic attack abilities must be purchasable in the shop between encounters.
  \item They must not be purchasable if the player does not have enough coins.
  \item They must reduce the amount of coins the player has by their cost when purchased.
  \item They must only be purchased once per run.
\end{itemize}

\paragraph{Enemies Can Use Attack Abilities}
\textbf{Value:} 3\quad \textbf{Priority:} Must\\
\textbf{Acceptance Criteria} 
\begin{itemize} 
  \item Enemies must be able to use both magic and mundane attack abilities on their turn against the player.
  \item Enemies must have the attack abilities based on their enemy type defined in the game design documentation.
  \item These abilities must deal damage of the type defined in the game design documentation.
  \item These abilities must deal an amount of damage based on the game design documentation, to the player. 
  \item The player's health must be reduced by this amount of damage. 
  \item An enemy whose health is 0 or less is dead and must not be able to use attack abilities.
\end{itemize}

\paragraph{Player Benefits from Purchased Damage Reistance Passive Abilities}
\textbf{Value:} 2\quad \textbf{Priority:} Could\\
\textbf{Acceptance Criteria} 
\begin{itemize} 
  \item The player must be able to purchase passive abilities in the shop that provide resistance to a specific damage type, as specified in the game design documentation.
  \item These abilities must affect the player in an encounter by halving the damage they take when the damage being dealt to them is that the upgrade provides resistance to.
  \item These must not be purchasable if the player does not have enough coins.
  \item These must reduce the amount of coins the player has by their cost when purchased.
  \item They must only be purchased once per run.
\end{itemize}

\paragraph{Player Benefits from Purchased Damage Boost Passive Abilities}
\textbf{Value:} 2\quad \textbf{Priority:} Could\\
\textbf{Acceptance Criteria} 
\begin{itemize} 
  \item The player must be able to purchase passive abilities in the shop that provide boosts to a specific damage type, as specified in the game design documentation.
  \item These abilities must affect the player in an encounter by doubling the damage they deal when the damage being dealt by them is that the upgrade boosts.
  \item These must not be purchasable if the player does not have enough coins.
  \item These must reduce the amount of coins the player has by their cost when purchased.
  \item They must only be purchased once per run. 
\end{itemize}

\paragraph{Enemies Have Damage Resistances and Vulnerabilities}
\textbf{Value:} 2\quad \textbf{Priority:} Could\\
\textbf{Acceptance Criteria} 
\begin{itemize} 
  \item Some enemies must resist a damage type as specified in the game design documentation.
  \item Some enemies must be weak to a damage type as specified in the game design documentation.
  \item When an enemy is dealt damage of a type they resist, that damage must be halved.
  \item When an enemy is dealt damage of a type they are vulnerable to, that damage must be doubled.
\end{itemize}

\paragraph{Players can Encounter Multiple Enemies in Combat}
\textbf{Value:} 2\quad \textbf{Priority:} Could\\
\textbf{Acceptance Criteria} 
\begin{itemize} 
  \item Combat encounters must be able to have multiple enemies within them.
  \item Each enemy must target the player with their attack abilities.
  \item The player must be able to chose which enemy to target with their attack abilities.
\end{itemize}

\paragraph{Players can Complete a Combat Round}
\textbf{Value:} 1\quad \textbf{Priority:} Must\\
\textbf{Acceptance Criteria} 
\begin{itemize} 
  \item At the start of a combat round, the player must gain magic based on the game design specification.
  \item They must then take their turn.
  \item Then all enemies must take their turns.
  \item Then the round is complete and starts from the beginning, granting the player magic.
\end{itemize}

\paragraph{Players can Complete an Encounter}
\textbf{Value:} 1\quad \textbf{Priority:} Must\\
\textbf{Acceptance Criteria} 
\begin{itemize} 
  \item A combat encounter must be completed if all enemies within the encounter have 0 or less health.
  \item When complete the player must gain coins based on the game design documentation.
\end{itemize}

\paragraph{Players can Retry an Encounter}
\textbf{Value:} 1\quad \textbf{Priority:} Must\\
\textbf{Acceptance Criteria} 
\begin{itemize} 
  \item At the start of a game run the player must have an amount of lives based on the design parameter.
  \item If the player's health reaches 0 or less in an encounter they must die.
  \item If a player dies while having 1 or more lives they must restart the same encounter. Their health and magic must be reset to their starting values, and as should the health of all enemies within the encounter. 
  \item If a player dies their lives must be reduced by 1.
\end{itemize}

\paragraph{Players can Progress Through Stages}
\textbf{Value:} 1\quad \textbf{Priority:} Must\\
\textbf{Acceptance Criteria} 
\begin{itemize} 
  \item After completing an encounter the player must go to the shop.
  \item After leaving the shop the stage the player is on must be incremented.
  \item The player must then enter another encounter.
\end{itemize}

\paragraph{Players can Fail Runs}
\textbf{Value:} 1\quad \textbf{Priority:} Must\\
\textbf{Acceptance Criteria} 
\begin{itemize} 
  \item If a player dies during an encounter, while having 0 or less lives they must fail the run.
  \item If a player fails the run, the run must end displaying a run over screen.
  \item The screen must show what stage the player reached, how long the run was, how many coins they had and how many times they died.
  \item The player must lose all their purchased abilities, and have their stage reset along with all encounters.
  \item The player must then be able to start a new run after leaving this screen.
\end{itemize}

\paragraph{Players can Complete Runs}
\textbf{Value:} 1\quad \textbf{Priority:} Must\\
\textbf{Acceptance Criteria} 
\begin{itemize} 
  \item If a player completes the encounter on stage 2 they must complete the run.
  \item This must show a run completed screen displaying what stage the player reached, how long the run was, how many coins they finished with and how many times they died.
  \item They must then be able to leave this screen and start a new run.
\end{itemize}

\subsubsection{Telemetry Events}

\paragraph{Application Handles Start Session}
\textbf{Value:} 1\quad \textbf{Priority:} Must\\
\textbf{Acceptance Criteria}
\begin{itemize}
  \item The game must feature the start session event.
  \item The event must have the fields specified in the telemetry schema.
  \item The event must be created when a new run starts.
\end{itemize}

\paragraph{Application Handles Normal Encounter Start}
\textbf{Value:} 1\quad \textbf{Priority:} Must\\
\textbf{Acceptance Criteria}
\begin{itemize}
  \item The game must feature the normal encounter start event.
  \item The event must have the fields specified in the telemetry schema.
  \item The event must be created each time a normal encounter starts, including restarts.
\end{itemize}

\paragraph{Application Handles Normal Encounter Fail}
\textbf{Value:} 1\quad \textbf{Priority:} Must\\
\textbf{Acceptance Criteria}
\begin{itemize}
  \item The game must feature the normal encounter fail event.
  \item The event must have the fields specified in the telemetry schema.
  \item The event must be created each time a normal encounter is failed by the user's character dying.
\end{itemize}

\paragraph{Application Handles Normal Encounter Complete}
\textbf{Value:} 1\quad \textbf{Priority:} Must\\
\textbf{Acceptance Criteria}
\begin{itemize}
  \item The game must feature the normal encounter complete event.
  \item The event must have the fields specified in the telemetry schema.
  \item The event must be created each time the player completes a normal encounter by killing all the enemies.
\end{itemize}

\paragraph{Application Handles Boss Encounter Start}
\textbf{Value:} 0\quad \textbf{Priority:} Wont\\
\textbf{Acceptance Criteria} N/A

\paragraph{Application Handles Boss Encounter Fail}
\textbf{Value:} 0\quad \textbf{Priority:} Wont\\
\textbf{Acceptance Criteria} N/A

\paragraph{Application Handles Boss Encounter Complete}
\textbf{Value:} 0\quad \textbf{Priority:} Wont\\
\textbf{Acceptance Criteria} N/A

\paragraph{Application Handles Gain Coin}
\textbf{Value:} 1\quad \textbf{Priority:} Could \\
\textbf{Acceptance Criteria} 
\begin{itemize}
  \item The game must feature the gain coin event.
  \item The event must have the fields specified in the telemetry schema.
  \item The event must be created when the player gains coins. 
\end{itemize}

\paragraph{Application Handles Buy Upgrade}
\textbf{Value:} 1\quad \textbf{Priority:} Could \\
\textbf{Acceptance Criteria} 
\begin{itemize}
  \item The game must feature the buy upgrade event.
  \item The event must have the fields specified in the telemetry schema.
  \item The event must be created when the player buys an ability. 
\end{itemize}

\paragraph{Application Handles End Session}
\textbf{Value:} 1\quad \textbf{Priority:} Must\\
\textbf{Acceptance Criteria} 
\begin{itemize}
  \item The game must feature the End session event.
  \item The event must have the fields specified in the telemetry schema.
  \item The event must be created when a game run ends, either by the user completing the run by completing the second encounter, or by their character dying.
\end{itemize}

\paragraph{Application Handles Settings Change}
\textbf{Value:} 1\quad \textbf{Priority:} Must\\
\textbf{Acceptance Criteria} 
\begin{itemize}
  \item The game must feature the settings change event.
  \item The event must have the fields specified in the telemetry schema.
  \item The event must be created when the user changes a setting's value and saves this change. 
\end{itemize}

\paragraph{Application Handles Kill Enemy}
\textbf{Value:} 1\quad \textbf{Priority:} Could \\
\begin{itemize}
  \item The game must feature the kill enemy event.
  \item The event must have the fields specified in the telemetry schema.
  \item The event must be created when the player kills an enemy. 
\end{itemize}