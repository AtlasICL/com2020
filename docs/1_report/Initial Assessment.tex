\section{Initial Assessment of Project}

As of the prototype phase, the code product is in a position where the majority of the elements of the initial vision are already realised.

This prototype displays the successful implementation of encounters and abilities within the game - each giving the player many opportunities to make interesting decisions that will influence their chances of success. As of this, the core idea of the gameplay system has been achieved, although this is somewhat limited, as the prototype only features two levels, one opportunity to spend coins, and a limited range of abilities, the full extent of the gameplay and its enjoyability cannot yet be fully stated, but within the confines of a single vertical slice, it is evident that this project has demonstrated its potential. On reflection, It’s evident that gameplay is currently lacking variety, there is not a broad enough selection of ways a player can improve their character, and there is not an engaging GUI. Moving forward from this prototype, these issues will be mitigated via the inclusion of more levels with more encounterable enemies at once, the introduction of abilities that boost a particular damage type, and a basic GUI.

Currently, a user who takes on the role of “Designer” can modify the starting player lives for each difficulty level, this decision is likely to be influenced by this user’s analysis of the telemetry interface, (potentially analysing that on average, players are using too many or too few lives). While this does provide a means for a designer to balance the difficulty of the game, it is apparent that there needs to be more parameters the designer can modify for the game to be able to be balanced in a nuanced and thoughtful way. The prototype already provides most of the internal framework to accommodate for more parameters to be modified, such as the amount of damage an enemy deals, or the price of an item in the game’s shop.

While the game is not yet in a complete state, the supporting telemetry interface is in a much more finalised state. With this implemented, a pseudoanonymous snapshot of player attributes (e.g. User and Session IDs, Timestamps, remaining lives) is recorded - for example, when a player starts a session, or completes an encounter. While the telemetry interface proves itself to be effective, the full extent of the utility cannot yet be fully displayed, as of the limited scope of the prototype. Once the main game has more opportunities to interact with the telemetry side, in the form of more encounters, the telemetry interface will be able to demonstrate its value in a broader context, with real player data. Unfortunatley, as of time constraints, not every single element of the telemetry interface supports alternate text, making some aspects inaccessible to those using screen readers, this may be something which will be fixed in later versions as to best follow the WCAG 2.1 principles.

To conclude, this prototype exhibits the most important elements of the final project, however, as it is still awaiting features, it cannot yet fully be considered a working solution for our problem. Yet, it has been designed in such a way that the framework is present for these currently unimplemented features. As a result, the prototype succeeds as a point at which the group’s collective effort can be viewed as a single end-product, and used as a stepping stone to a final project that both meets our initial vision, but takes into account the issues in the prototype (i.e, a lack of ability variety). Overall, this project succeeds in providing a clear and simple vertical slice of the developing project, which is on track to meet its aims.
