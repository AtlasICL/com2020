
\subsection{Documentation}
\subsubsection{Prototype Report}
\paragraph{Executive Summary}
\textbf{Value:} 2 \quad \textbf{Priority:} Must\\
\textbf{Acceptance Criteria}
\begin{itemize} 
    \item The summary appears at the start of the report.
    \item The summary clearly states what is delivered in the prototype.
    \item The summary clearly states why prototype features exist in the prototype. 
    \item The summary clearly states the most important risks of the prototype.
    \item The summary clearly states the mitigations for those risks.
\end{itemize}
\paragraph{Problem Framing}
\textbf{Value:} 3\quad \textbf{Priority:} Must\\
\textbf{Acceptance Criteria}
\begin{itemize} 
    \item The problem framing clearly explains the problem.
    \item The problem framing clearly states why the problem is important.
    \item The problem framing clearly states what the goal of the project is.
    \item The problem framing must state who is affected by the problem. 
\end{itemize}

\paragraph{Project Backlog}
\textbf{Value:} 3\quad \textbf{Priority:} Must\\
\textbf{Acceptance Criteria}
\begin{itemize} 
    \item The project backlog must clearly state which epics are required to complete the project.
    \item The project backlog must clearly state which user stories are associated with each epic.
\end{itemize}
\paragraph{Sprint One Prioritised Requirements}
\textbf{Value:} 3\quad \textbf{Priority:} Must\\
\textbf{Acceptance Criteria}
\begin{itemize} 
  \item Each epic in the project backlog must be stated in the requirements.
  \item Each story within each epic must be stated in the requirements.
  \item Each story should have a value associated with it, based on how much work it is to complete the story, the significance of the risks associated with it, the technical changes required to complete it, how much depends on it, how complex it is to complete, and the value it's completion adds to the project.
  \item Each story must have an acceptance criteria associated with it, which when all criteria are met the story is considered complete for sprint 1. Criteria may change between sprints as sprint requirements change. 
  \item Each story must have a priority associated with it for that sprint. The priority should either be Must, Should, Could or Won't. 
  \item Each story must be able to link to a prototype feature and test. 
\end{itemize}
\paragraph{Architecture Schema}
\textbf{Value:} 3\quad \textbf{Priority:} Should\\
\textbf{Acceptance Criteria}
\begin{itemize} 
  \item Each class within the Java application must have it's purpose and use case explained.
  \item There must be a UML diagram of the Java application showing how classes interact.
  \item Each component of the Python application must have it's purpose and use case explained. 
  \item All design choices must be justified and explained.
  \item The limitations of all design choices must be stated, and why they're acceptable must be explained.
\end{itemize}
\paragraph{Data Flow Schema}
\textbf{Value:} 2\quad \textbf{Priority:} Should\\
\textbf{Acceptance Criteria}
\begin{itemize} 
  \item Each component that accesses data should be explained, describing what data it accesses, what it does with it, and why it needs to access this data.
  \item Each data store should be explained, stating what data it stores and in what format. 
\end{itemize}
\paragraph{Telemetry Schema}
\textbf{Value:} 3\quad \textbf{Priority:} Must\\
\textbf{Acceptance Criteria}
\begin{itemize} 
  \item Each telemetry event must be described by the schema.
  \item Each telemetry event must have all it's fields described by the schema.
  \item Each field must have its domain described by the schema. For enumerated fields all valid values must be described.
\end{itemize}
\paragraph{Initial Evaluation}
\textbf{Value:} 5\quad \textbf{Priority:} Must\\
\textbf{Acceptance Criteria}
\begin{itemize} 
  \item Several metrics and measures of success should be evaluated. Where applicable they should be compared with the baselines of other solutions to the problem.
  \item The method coverage of the test suite should be evaluated to determine how thoroughly the system is examined, and thus how confident we can be that defects will be discovered early. 
  \item The time between the user inputting what attack ability they use, and the command line outputting the result of their attack and the enemies' turn to measure the responsiveness of the game. 
  \item The memory usage of the prototype must be less than 1GB, and ideally as small as possible, to ensure it can run on devices with small amounts of memory. 
  \item The ability for the prototype to handle invalid input gracefully as to make it accessible and secure.
  \item The ability for the prototype to handle exceptions and edge cases gracefully to ensure the user experience is smooth and error free.
  \item The ability for the prototype to handle anomalous telemetry events gracefully to ensure the designer experience is smooth and error free.
  \item The usability of the prototype to allow designers to locate issues with the game and make informed changes within a short timeframe to ensure it is a useful tool for designers.
  \item The effectiveness of the rule based suggestion to ensure they provide help to designers.
  \item The limitations of the evaluation, including biases in the methodology, and whether the tests and evaluation methods generalise such that they're comparable to real world scenarios. 
\end{itemize}
\paragraph{Sprint Two Prioritised Requirements}
\textbf{Value:} 3\quad \textbf{Priority:} Should\\
\textbf{Acceptance Criteria}
\begin{itemize}
  \item Each epic in the project backlog must be stated in the requirements.
  \item Each story within each epic must be stated in the requirements.
  \item Each story should have a value associated with it, based on how much work it is to complete the story, the significance of the risks associated with it, the technical changes required to complete it, how much depends on it, how complex it is to complete, and the value it's completion adds to the project.
  \item Each story must have an acceptance criteria associated with it, which when all criteria are met the story is considered complete for sprint 2. Criteria may change between sprints as sprint requirements change. 
  \item Each story must have a priority associated with it for that sprint. The priority should either be Must, Should, Could or Won't. 
  \item Each story must be able to link to a final product feature and test. 
\end{itemize}
\paragraph{Management}
\textbf{Value:} 2\quad \textbf{Priority:} Should\\
\textbf{Acceptance Criteria}
\begin{itemize}
  \item The management section should outline what roles each member had.
  \item It should outline what tasks members owned, and what tasks they contributed to.
  \item It should outline in what areas the group worked well together.
  \item It should outline what challenges the group faced working together.
  \item It should outline how those challenges were solved. 
\end{itemize}
\subsubsection{Meeting Minutes}
\paragraph{Meeting Attendance}
\textbf{Value:} 1\quad \textbf{Priority:} Must\\
\textbf{Acceptance Criteria}
\begin{itemize}
  \item For each meeting, the group members who did attend the meeting are noted.
  \item For each meeting, the group members who did not attend the meeting are noted. 
\end{itemize}
\paragraph{Work Completed}
\textbf{Value:} 1\quad \textbf{Priority:} Must\\
\textbf{Acceptance Criteria}
\begin{itemize}
  \item For each meeting, if a group member was present the work they completed during the previous weekly scrum should be outlined.
\end{itemize}
\paragraph{Topics Discussed}
\textbf{Value:} 1\quad \textbf{Priority:} Must\\
\textbf{Acceptance Criteria}
\begin{itemize}
  \item For each meeting, what topics were presented by group members and discussed by the group must be recorded.
\end{itemize}
\paragraph{Work to Be Completed}
\textbf{Value:} 1\quad \textbf{Priority:} Must\\
\textbf{Acceptance Criteria}
\begin{itemize}
  \item For each meeting, the work assigned to each group member (whether they were there or not) must be recorded.
\end{itemize}
\subsubsection{Risk Register}
\paragraph{List of Risks}
\textbf{Value:} 2\quad \textbf{Priority:} Must\\
\textbf{Acceptance Criteria}
\begin{itemize}
  \item A list of all risks associated with role based access control should be included.
  \item A list of all risks associated with authentication should be included.
  \item A list of all risks associated with the storage of telemetry data should be included.
  \item A list of all risks associated with design parameters should be included.
  \item A list of all risks associated with logging should be included.
  \item A list of all risks associated with the storage of user information should be included.
\end{itemize}
\paragraph{Risk Mitigations}
\textbf{Value:} 2\quad \textbf{Priority:} Must\\
\textbf{Acceptance Criteria}
\begin{itemize}
  \item For each risk, the mitigation strategy in use should be described and explained.
\end{itemize}
\subsubsection{Ethical and Legal Considerations}
\paragraph{Privacy and Data Protection Analysis}
\textbf{Value:} 1\quad \textbf{Priority:} Must\\
\textbf{Acceptance Criteria}
\begin{itemize}
  \item Must discuss what data is stored about users.
  \item Must discuss how the data storage complies with law.
  \item Must discuss how collected data is pseudonymised.
  \item Must discuss how access to this data is limited to only those it's relevant to.
\end{itemize}
\paragraph{Consent and Disclosure Analysis}
\textbf{Value:} 1\quad \textbf{Priority:} Must\\
\textbf{Acceptance Criteria}
\begin{itemize}
  \item Must discuss what the data is used for.
  \item Must discuss how consent is handled.
  \item Must discuss how this consent and disclosure policy complies with law.
  \item Must discuss how this policy affects the user experience. 
\end{itemize}
\paragraph{Accessabilty Analysis}
\textbf{Value:} 1\quad \textbf{Priority:} Should\\
\textbf{Acceptance Criteria}
\begin{itemize}
  \item Must discuss the importance of accessability.
  \item Must discuss what systems are in place to support the accessability of the software.
  \item Must discuss what areas the system can improve upon for sprint 2.
\end{itemize}
\paragraph{Intelectual Property and Licensing Implications}
\textbf{Value:} 1\quad \textbf{Priority:} Must\\
\textbf{Acceptance Criteria}
\begin{itemize}
  \item Must discuss the intellectual property rights of the developers.
  \item Must discuss the intellectual property rights of the product owner.
  \item Must discuss the licensing considerations for the software and data the system depends on.
  \item Must discuss the intellectual property rights of sources the systems draws inspiration from or is similar to.
\end{itemize}

\subsubsection{Project License}
\paragraph{Determine the Project License}
\textbf{Value:} 1\quad \textbf{Priority:} Must\\
\textbf{Acceptance Criteria}
\begin{itemize}
  \item A license for the project must be determined.
  \item The license must comply with the licensing requirements of all dependencies. 
\end{itemize}
\subsubsection{Software and Data Inventory}
\paragraph{Software Inventory}
\textbf{Value:} 2\quad \textbf{Priority:} Should\\
\textbf{Acceptance Criteria}
\begin{itemize}
  \item An inventory of all software components the system directly depends on must be produced.
  \item This must contain the license of each dependency.
  \item This must contain the cost model of each dependency.
  \item This must contain the provenance of each dependency.
  \item This must contain the version of each dependency the system uses.
\end{itemize}
\paragraph{Data Inventory}
\textbf{Value:} 1\quad \textbf{Priority:} Should\\
\textbf{Acceptance Criteria}
\begin{itemize}
  \item An inventory of all data the system directly depends on or uses must be produced.
  \item This must contain the license of each dependency.
  \item This must contain the cost model of each dependency.
  \item This must contain the provenance of each dependency.
  \item This must contain the version of each dependency the system uses.
\end{itemize}
\subsubsection{Deployment Guide}
\paragraph{Java Application Deployment Instructions}
\textbf{Value:} 1\quad \textbf{Priority:} Must\\
\textbf{Acceptance Criteria}
\begin{itemize}
  \item The guide must contain clear instructions for how to setup and run the Java application.
  \item There may be instructions for different systems, but there should be a way to run the application on any desktop system that runs Windows, MacOS or Linux.
\end{itemize}
\paragraph{Python Application Deployment Instructions}
\textbf{Value:} 1\quad \textbf{Priority:} Must\\
\textbf{Acceptance Criteria}
\begin{itemize}
  \item The guide must contain clear instructions for how to setup and run the Python application.
  \item There may be instructions for different systems, but there should be a way to run the application on any desktop system that runs Windows, MacOS or Linux.
\end{itemize}
\paragraph{Automated Test Running Instructions}
\textbf{Value:} 1\quad \textbf{Priority:} Must\\
\textbf{Acceptance Criteria}
\begin{itemize}
  \item There should be clear instructions on how to run the test suite on the source code.
  \item The version of the testing framework should be provided.
  \item The running instructions should allow the running of the test on any desktop system that runs either Windows, MacOS or Linux.
\end{itemize}
\subsubsection{Test Evidence}
\paragraph{Automated Test Evidence}
\textbf{Value:} 2\quad \textbf{Priority:} Must\\
\textbf{Acceptance Criteria}
\begin{itemize}
  \item There must be 5 automated tests.
  \item For each automated test it should be outlined what component is being tested.
  \item For each automated test it should be outlined what the expected output is.
  \item For each automated test the actual output of the system should be provided, and thus evidence that the system passes the test.
  \item The system must pass all automated tests.
\end{itemize}
\paragraph{End to End Test Evidence}
\textbf{Value:} 1\quad \textbf{Priority:} Must\\
\textbf{Acceptance Criteria}
\begin{itemize}
  \item There must be one end to end test.
  \item Evidence that the test is successful should be given.
  \item The happy path and failure cases for the the test must be described.
  \item The expected results of the test should be described.
\end{itemize}
\subsubsection{Presentation}
\paragraph{The Problem}
\textbf{Value:} 2\quad \textbf{Priority:} Must\\
\textbf{Acceptance Criteria}
\begin{itemize}
  \item Should outline what the problem is.
  \item Should outline why the problem is relevant.
  \item Should outlie the scope of the problem.
  \item Should outline the scope of the solution.
  \item Should outline what assumptions were made about the problem.
  \item Should outline who the intended users of the solution are. 
\end{itemize}
\paragraph{Approach}
\textbf{Value:} 2\quad \textbf{Priority:} Must\\
\textbf{Acceptance Criteria}
\begin{itemize}
  \item Should outline the architecture of the solution.
  \item Should speak about similar solutions.
  \item Should outline measures used to evaluate the system.
\end{itemize}
\paragraph{Implementation}
\textbf{Value:} 2\quad \textbf{Priority:} Must\\
\textbf{Acceptance Criteria}
\begin{itemize}
  \item Should outline the key components of the solution.
  \item Should outline the key features of the implementation.
  \item Should outline why key design decisions were made.
\end{itemize}
\paragraph{Evaluation}
\textbf{Value:} 2\quad \textbf{Priority:} Must\\
\textbf{Acceptance Criteria}
\begin{itemize}
  \item Should outline what issues were faced during implementation.
  \item Should outline how well the system performs against measures outlined in approach, and compare it to existing solutions.
  \item Should provide clear evidence of how well the prototype works. 
\end{itemize}
\paragraph{Demonstration}
\textbf{Value:} 3\quad \textbf{Priority:} Must\\
\textbf{Acceptance Criteria}
\begin{itemize}
  \item Should take around 3 minutes to complete.
  \item Should demonstrate the core functionality of the system.
  \item Should be rehearsed before the live demo, to ensure it works correctly.
  \item Should have a backup demonstration/video if the demo goes wrong. 
\end{itemize}
\paragraph{Limitations and Next Steps}
\textbf{Value:} 2\quad \textbf{Priority:} Must\\
\textbf{Acceptance Criteria}
\begin{itemize}
  \item Should outline the areas in which the prototype is weakest.
  \item Should briefly outline what is to be done in the next sprint.
\end{itemize}
\subsubsection{Operation Guide}
\paragraph{Operation Instructions}
\textbf{Value:} 0\quad \textbf{Priority:} Won't\\
\textbf{Acceptance Criteria} N/A
\paragraph{Maintenance Instructions}
\textbf{Value:} 0\quad \textbf{Priority:} Won't\\
\textbf{Acceptance Criteria} N/A
\paragraph{Troubleshooting Instructions}
\textbf{Value:} 0\quad \textbf{Priority:} Won't\\
\textbf{Acceptance Criteria} N/A
\paragraph{Extension Guide}
\textbf{Value:} 0\quad \textbf{Priority:} Won't\\
\textbf{Acceptance Criteria} N/A
\subsubsection{Data Management Guide}
\paragraph{Data Stored}
\textbf{Value:} 0\quad \textbf{Priority:} Won't\\
\textbf{Acceptance Criteria} N/A
\paragraph{Data Format}
\textbf{Value:} 0\quad \textbf{Priority:} Won't\\
\textbf{Acceptance Criteria} N/A
\subsubsection{Scrum Board}
\paragraph{Backlog, In Progress and Done Sections}
\textbf{Value:} 3\quad \textbf{Priority:} Should\\
\textbf{Acceptance Criteria} 
\begin{itemize}
  \item There must be a card for each story in the project backlog defined in this report.
  \item There must be a card for each group of acceptance criteria (task) that varies between sprints.
  \item There must be epics created on the board for each epic defined in the project backlog in this report.
  \item There must be a backlog section for tasks that have yet to have been started.
  \item There must be an in progress section for tasks being worked on. Each task in this section must be assigned to at least one group member.
  \item THere must be a done section for tasks that are complete. Each task in this section must be assigned to at least one group member.
\end{itemize}
\subsection{Python Application}
\subsubsection{User Interface}
\paragraph{Login}

\paragraph{Dashboards}

\paragraph{Suggestions}

\paragraph{Decision Log}

\subsubsection{Authentication and Acess Control}
\paragraph{Login}

\paragraph{Password Reset}

\paragraph{Player Permissions}

\paragraph{Designer Permissions}

\paragraph{Developer Permissions}

\subsubsection{Dashboard Views}
\paragraph{Simulation View}

\paragraph{Funnel View}

\paragraph{Difficulty Spikes}

\paragraph{Progress Curves}

\paragraph{Fairness Indicators}

\paragraph{Comparison Mode}

\subsubsection{Design Suggestions}
\paragraph{Rule Based Design Suggestions}

\subsubsection{Telemetry Events}
\paragraph{Read Telemetry Events}

\paragraph{Validate Telemetry Events}

\subsubsection{Seeded Dataset}
\paragraph{Telemetry Events}

\paragraph{Stage Variety}

\paragraph{Difficulty Variety}

\paragraph{User Variety}

\paragraph{Session Variet}

\paragraph{Anomalous Telemetry Events}

\paragraph{Balancing Decisions}

\subsection{Java Application}
\subsubsection{User Interface}
\paragraph{Login}

\paragraph{Main Menu}

\paragraph{Settings}

\paragraph{Start Run}

\paragraph{Encounter}

\paragraph{Shop}

\paragraph{End Run}

\subsubsection{Autheentication and Acesss Control}
\paragraph{Login}

\paragraph{Login Reset}

\paragraph{Player Permissions}

\paragraph{Designer Permissions}

\paragraph{Develoer Permissions}

\subsubsection{Settings}
\paragraph{Telemetry}

\paragraph{Design Parameters}

\paragraph{Simulation Mode}

\paragraph{Role Asignment}

\subsubsection{Gameplay Loop}
\paragraph{Start Run}

\paragraph{Chose Difficulty}

\paragraph{Be in an Encounter}

\paragraph{Use Physcal Attack Abilities}

\paragraph{Use Magical Attack Abilities}

\paragraph{Take Turns}

\paragraph{Kill Enemies}

\paragraph{Complete an Encounter}

\paragraph{View Shop}

\paragraph{Purchase Upgrades in the Shop}

\paragraph{Complete a Run}

\subsubsection{Game Mechanics}
\paragraph{Coins}

\paragraph{Health}

\paragraph{Lives}

\paragraph{Damage Types}

\paragraph{Magic}

\paragraph{Passive Abilities}

\paragraph{Attack Abilities}

\paragraph{Difficulties}

\subsubsection{Telemetry Events}
\paragraph{Event Writing}

\paragraph{Event Validation}

\paragraph{Start Session}

\paragraph{Normal Encounter Start}

\paragraph{Normal Encounter Fail}

\paragraph{Normal Encounter Complete}

\paragraph{Boss Encounter Start}

\paragraph{Boss Encounter Fail}

\paragraph{Boss Encounter Complete}

\paragraph{Gain Coin}

\paragraph{Buy Upgrade}

\paragraph{End Session}

\paragraph{Settings Change}

\paragraph{Kill Enemy}
