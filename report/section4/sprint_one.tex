\subsection{Documentation}
\subsubsection{Report}
\paragraph{Executive Summary}
\textbf{Value:} 2 \quad \textbf{Priority:} Must\\
\textbf{Acceptance Criteria}
\begin{itemize} 
    \item The summary appears at the start of the report.
    \item The summary clearly states what is delivered in the prototype.
    \item The summary clearly states why prototype features exist in the prototype. 
    \item The summary clearly states the most important risks of the prototype.
    \item The summary clearly states the mitigations for those risks.
\end{itemize}
\paragraph{Problem Framing}
\textbf{Value:} 3\quad \textbf{Priority:} Must\\
\textbf{Acceptance Criteria}
\begin{itemize} 
    \item The problem framing clearly explains the problem.
    \item The problem framing clearly states why the problem is important.
    \item The problem framing clearly states what the goal of the project is.
    \item The problem framing must state who is affected by the problem. 
\end{itemize}
\paragraph{Project Backlog}
\textbf{Value:} 3\quad \textbf{Priority:} Must\\
\textbf{Acceptance Criteria}
\begin{itemize} 
    \item The project backlog must clearly state which epics are required to complete the project.
    \item The project backlog must clearly state which user stories are associated with each epic.
\end{itemize}
\paragraph{Sprint One Prioritised Requirements}
\textbf{Value:} 3\quad \textbf{Priority:} Must\\
\textbf{Acceptance Criteria}
\begin{itemize} 
  \item Each epic in the project backlog must be stated in the requirements.
  \item Each story within each epic must be stated in the requirements.
  \item Each story should have a value associated with it, based on how much work it is to complete the story, the significance of the risks associated with it, the technical changes required to complete it, how much depends on it, how complex it is to complete, and the value it's completion adds to the project.
  \item Each story must have an acceptance criteria associated with it, which when all criteria are met the story is considered complete for sprint 1. Criteria may change between sprints as sprint requirements change. 
  \item Each story must have a priority associated with it for that sprint. The priority should either be Must, Should, Could or Wont. 
  \item Each story must be able to link to a prototype feature and test. 
\end{itemize}
\paragraph{Architecture Schema}
\textbf{Value:} 3\quad \textbf{Priority:} Should\\
\textbf{Acceptance Criteria}
\begin{itemize} 
  \item Each class within the Java application must have it's purpose and use case explained.
  \item There must be a UML diagram of the Java application showing how classes interact.
  \item Each component of the Python application must have it's purpose and use case explained. 
  \item All design choices must be justified and explained.
  \item The limitations of all design choices must be stated, and why they're acceptable must be explained.
\end{itemize}
\paragraph{Data Flow Schema}
\textbf{Value:} 2\quad \textbf{Priority:} Should\\
\textbf{Acceptance Criteria}
\begin{itemize} 
  \item Each component that accesses data should be explained, describing what data it accesses, what it does with it, and why it needs to access this data.
  \item Each data store should be explained, stating what data it stores and in what format. 
\end{itemize}
\paragraph{Telemetry Schema}
\textbf{Value:} 3\quad \textbf{Priority:} Must\\
\textbf{Acceptance Criteria}
\begin{itemize} 
  \item Each telemetry event must be described by the schema.
  \item Each telemetry event must have all it's fields described by the schema.
  \item Each field must have its domain described by the schema. For enumerated fields all valid values must be described.
\end{itemize}
\paragraph{Evaluation}
\textbf{Value:} 5\quad \textbf{Priority:} Must\\
\textbf{Acceptance Criteria}
\begin{itemize} 
  \item Several metrics and measures of success should be evaluated. Where applicable they should be compared with the baselines of other solutions to the problem.
  \item Each metric and measure should serve an explicit purpose, which must be explained.
  \item The method coverage of the test suite should be evaluated to determine how thoroughly the system is examined, and thus how confident we can be that defects will be discovered early. 
  \item The time between the user inputting what attack ability they use, and the command line outputting the result of their attack and the enemies' turn to measure the responsiveness of the game. 
  \item The memory usage of the prototype must be less than 1GB, and ideally as small as possible, to ensure it can run on devices with small amounts of memory. 
  \item The ability for the prototype to handle invalid input gracefully as to make it accessible and secure.
  \item The ability for the prototype to handle exceptions and edge cases gracefully to ensure the user experience is smooth and error free.
  \item The ability for the prototype to handle anomalous telemetry events gracefully to ensure the designer experience is smooth and error free.
  \item The usability of the prototype to allow designers to locate issues with the game and make informed changes within a short timeframe to ensure it is a useful tool for designers.
  \item The effectiveness of the rule based suggestion to ensure they provide help to designers.
  \item The limitations of the evaluation, including biases in the methodology, and whether the tests and evaluation methods generalise such that they're comparable to real world scenarios. 
\end{itemize}
\paragraph{Sprint Two Prioritised Requirements}
\textbf{Value:} 3\quad \textbf{Priority:} Should\\
\textbf{Acceptance Criteria}
\begin{itemize}
  \item Each epic in the project backlog must be stated in the requirements.
  \item Each story within each epic must be stated in the requirements.
  \item Each story should have a value associated with it, based on how much work it is to complete the story, the significance of the risks associated with it, the technical changes required to complete it, how much depends on it, how complex it is to complete, and the value it's completion adds to the project.
  \item Each story must have an acceptance criteria associated with it, which when all criteria are met the story is considered complete for sprint 2. Criteria may change between sprints as sprint requirements change. 
  \item Each story must have a priority associated with it for that sprint. The priority should either be Must, Should, Could or Wont. 
  \item Each story must be able to link to a final product feature and test. 
\end{itemize}
\paragraph{Management}
\textbf{Value:} 2\quad \textbf{Priority:} Should\\
\textbf{Acceptance Criteria}
\begin{itemize}
  \item The management section should outline what roles each member had.
  \item It should outline what tasks members owned, and what tasks they contributed to.
  \item It should outline in what areas the group worked well together.
  \item It should outline what challenges the group faced working together.
  \item It should outline how those challenges were solved. 
\end{itemize}
\subsubsection{Meeting Minutes}
\paragraph{Meeting Attendance}
\textbf{Value:} 1\quad \textbf{Priority:} Must\\
\textbf{Acceptance Criteria}
\begin{itemize}
  \item For each meeting, the group members who did attend the meeting are noted.
  \item For each meeting, the group members who did not attend the meeting are noted. 
\end{itemize}
\paragraph{Work Completed}
\textbf{Value:} 1\quad \textbf{Priority:} Must\\
\textbf{Acceptance Criteria}
\begin{itemize}
  \item For each meeting, if a group member was present the work they completed during the previous weekly scrum should be outlined.
\end{itemize}
\paragraph{Topics Discussed}
\textbf{Value:} 1\quad \textbf{Priority:} Must\\
\textbf{Acceptance Criteria}
\begin{itemize}
  \item For each meeting, what topics were presented by group members and discussed by the group must be recorded.
\end{itemize}
\paragraph{Work to Be Completed}
\textbf{Value:} 1\quad \textbf{Priority:} Must\\
\textbf{Acceptance Criteria}
\begin{itemize}
  \item For each meeting, the work assigned to each group member (whether they were there or not) must be recorded.
\end{itemize}
\subsubsection{Risk Register}
\paragraph{List of Risks}
\textbf{Value:} 2\quad \textbf{Priority:} Must\\
\textbf{Acceptance Criteria}
\begin{itemize}
  \item A list of all risks associated with role based access control should be included.
  \item A list of all risks associated with authentication should be included.
  \item A list of all risks associated with the storage of telemetry data should be included.
  \item A list of all risks associated with design parameters should be included.
  \item A list of all risks associated with logging should be included.
  \item A list of all risks associated with the storage of user information should be included.
\end{itemize}
\paragraph{Risk Mitigations}
\textbf{Value:} 2\quad \textbf{Priority:} Must\\
\textbf{Acceptance Criteria}
\begin{itemize}
  \item For each risk, the mitigation strategy in use should be described and explained.
\end{itemize}
\subsubsection{Ethical and Legal Considerations}
\paragraph{Privacy and Data Protection Analysis}
\textbf{Value:} 1\quad \textbf{Priority:} Must\\
\textbf{Acceptance Criteria}
\begin{itemize}
  \item Must discuss what data is stored about users.
  \item Must discuss how the data storage complies with law.
  \item Must discuss how collected data is pseudonymised.
  \item Must discuss how access to this data is limited to only those it's relevant to.
\end{itemize}
\paragraph{Consent and Disclosure Analysis}
\textbf{Value:} 1\quad \textbf{Priority:} Must\\
\textbf{Acceptance Criteria}
\begin{itemize}
  \item Must discuss what the data is used for.
  \item Must discuss how consent is handled.
  \item Must discuss how this consent and disclosure policy complies with law.
  \item Must discuss how this policy affects the user experience. 
\end{itemize}
\paragraph{Accessabilty Analysis}
\textbf{Value:} 1\quad \textbf{Priority:} Should\\
\textbf{Acceptance Criteria}
\begin{itemize}
  \item Must discuss the importance of accessability.
  \item Must discuss what systems are in place to support the accessability of the software.
  \item Must discuss what areas the system can improve upon for sprint 2.
\end{itemize}
\paragraph{Intelectual Property and Licensing Implications}
\textbf{Value:} 1\quad \textbf{Priority:} Must\\
\textbf{Acceptance Criteria}
\begin{itemize}
  \item Must discuss the intellectual property rights of the developers.
  \item Must discuss the intellectual property rights of the product owner.
  \item Must discuss the licensing considerations for the software and data the system depends on.
  \item Must discuss the intellectual property rights of sources the systems draws inspiration from or is similar to.
\end{itemize}

\subsubsection{Project License}
\paragraph{Determine the Project License}
\textbf{Value:} 1\quad \textbf{Priority:} Must\\
\textbf{Acceptance Criteria}
\begin{itemize}
  \item A license for the project must be determined.
  \item The license must comply with the licensing requirements of all dependencies. 
\end{itemize}
\subsubsection{Software and Data Inventory}
\paragraph{Software Inventory}
\textbf{Value:} 2\quad \textbf{Priority:} Should\\
\textbf{Acceptance Criteria}
\begin{itemize}
  \item An inventory of all software components the system directly depends on must be produced.
  \item This must contain the license of each dependency.
  \item This must contain the cost model of each dependency.
  \item This must contain the provenance of each dependency.
  \item This must contain the version of each dependency the system uses.
\end{itemize}
\paragraph{Data Inventory}
\textbf{Value:} 1\quad \textbf{Priority:} Should\\
\textbf{Acceptance Criteria}
\begin{itemize}
  \item An inventory of all data the system directly depends on or uses must be produced.
  \item This must contain the license of each dependency.
  \item This must contain the cost model of each dependency.
  \item This must contain the provenance of each dependency.
  \item This must contain the version of each dependency the system uses.
\end{itemize}
\subsubsection{Deployment and Operations Guide}
\paragraph{Java Application Deployment Instructions}
\textbf{Value:} 1\quad \textbf{Priority:} Must\\
\textbf{Acceptance Criteria}
\begin{itemize}
  \item The guide must contain clear instructions for how to setup and run the Java application.
  \item There may be instructions for different systems, but there should be a way to run the application on any desktop system that runs Windows, MacOS or Linux.
\end{itemize}
\paragraph{Python Application Deployment Instructions}
\textbf{Value:} 1\quad \textbf{Priority:} Must\\
\textbf{Acceptance Criteria}
\begin{itemize}
  \item The guide must contain clear instructions for how to setup and run the Python application.
  \item There may be instructions for different systems, but there should be a way to run the application on any desktop system that runs Windows, MacOS or Linux.
\end{itemize}
\paragraph{Automated Test Running Instructions}
\textbf{Value:} 1\quad \textbf{Priority:} Must\\
\textbf{Acceptance Criteria}
\begin{itemize}
  \item There should be clear instructions on how to run the test suite on the source code.
  \item The version of the testing framework should be provided.
  \item The running instructions should allow the running of the test on any desktop system that runs either Windows, MacOS or Linux.
\end{itemize}
\subsubsection{Test Evidence}
\paragraph{Automated Test Evidence}
\textbf{Value:} 2\quad \textbf{Priority:} Must\\
\textbf{Acceptance Criteria}
\begin{itemize}
  \item There must be 5 automated tests.
  \item For each automated test it should be outlined what component is being tested.
  \item For each automated test it should be outlined what the expected output is.
  \item For each automated test the actual output of the system should be provided, and thus evidence that the system passes the test.
  \item The system must pass all automated tests.
\end{itemize}
\paragraph{End to End Test Evidence}
\textbf{Value:} 1\quad \textbf{Priority:} Must\\
\textbf{Acceptance Criteria}
\begin{itemize}
  \item There must be one end to end test.
  \item Evidence that the test is successful should be given.
  \item The happy path and failure cases for the test must be described.
  \item The expected results of the test should be described.
\end{itemize}
\subsubsection{Presentation}
\paragraph{The Problem}
\textbf{Value:} 2\quad \textbf{Priority:} Must\\
\textbf{Acceptance Criteria}
\begin{itemize}
  \item Should outline what the problem is.
  \item Should outline why the problem is relevant.
  \item Should outlie the scope of the problem.
  \item Should outline the scope of the solution.
  \item Should outline what assumptions were made about the problem.
  \item Should outline who the intended users of the solution are. 
\end{itemize}
\paragraph{Approach}
\textbf{Value:} 2\quad \textbf{Priority:} Must\\
\textbf{Acceptance Criteria}
\begin{itemize}
  \item Should outline the architecture of the solution.
  \item Should speak about similar solutions.
  \item Should outline measures used to evaluate the system.
\end{itemize}
\paragraph{Implementation}
\textbf{Value:} 2\quad \textbf{Priority:} Must\\
\textbf{Acceptance Criteria}
\begin{itemize}
  \item Should outline the key components of the solution.
  \item Should outline the key features of the implementation.
  \item Should outline why key design decisions were made.
\end{itemize}
\paragraph{Evaluation}
\textbf{Value:} 2\quad \textbf{Priority:} Must\\
\textbf{Acceptance Criteria}
\begin{itemize}
  \item Should outline what issues were faced during implementation.
  \item Should outline how well the system performs against measures outlined in approach, and compare it to existing solutions.
  \item Should provide clear evidence of how well the prototype works. 
\end{itemize}
\paragraph{Demonstration}
\textbf{Value:} 3\quad \textbf{Priority:} Must\\
\textbf{Acceptance Criteria}
\begin{itemize}
  \item Should take around 3 minutes to complete.
  \item Should demonstrate the core functionality of the system.
  \item Should be rehearsed before the live demo, to ensure it works correctly.
  \item Should have a backup demonstration/video if the demo goes wrong. 
\end{itemize}
\paragraph{Limitations and Next Steps}
\textbf{Value:} 2\quad \textbf{Priority:} Must\\
\textbf{Acceptance Criteria}
\begin{itemize}
  \item Should outline the areas in which the prototype is weakest.
  \item Should briefly outline what is to be done in the next sprint.
\end{itemize}
\subsubsection{Maintainance and Troubleshooting Guide}
\paragraph{Maintenance Instructions}
\textbf{Value:} 0\quad \textbf{Priority:} Wont\\
\textbf{Acceptance Criteria} N/A
\paragraph{Troubleshooting Instructions}
\textbf{Value:} 0\quad \textbf{Priority:} Wont\\
\textbf{Acceptance Criteria} N/A
\paragraph{Extension Instructions}
\textbf{Value:} 0\quad \textbf{Priority:} Wont\\
\textbf{Acceptance Criteria} N/A
\subsubsection{Data Management Guide}
\paragraph{Data Stored}
\textbf{Value:} 0\quad \textbf{Priority:} Wont\\
\textbf{Acceptance Criteria} N/A
\paragraph{Data Format}
\textbf{Value:} 0\quad \textbf{Priority:} Wont\\
\textbf{Acceptance Criteria} N/A
\subsubsection{Scrum Board}
\paragraph{Backlog, In Progress and Done Sections}
\textbf{Value:} 3\quad \textbf{Priority:} Should\\
\textbf{Acceptance Criteria} 
\begin{itemize}
  \item There must be a card for each story in the project backlog defined in this report.
  \item There must be a card for each group of acceptance criteria (task) that varies between sprints.
  \item There must be epics created on the board for each epic defined in the project backlog in this report.
  \item There must be a backlog section for tasks that have yet to have been started.
  \item There must be an in progress section for tasks being worked on. Each task in this section must be assigned to at least one group member.
  \item There must be a done section for tasks that are complete. Each task in this section must be assigned to at least one group member.
\end{itemize}
\subsection{Python Application}
\subsubsection{User Interface}
\paragraph{Login}
\textbf{Value:} 2\quad \textbf{Priority:} Must\\
\textbf{Acceptance Criteria} 
\begin{itemize}
  \item Users of the python application should first see a login screen.
  \item This screen must prompt the users to sign in before they can access any other screen.
  \item Users should also be able to reset their login or create an account from this screen.
\end{itemize}
\paragraph{Dashboards}
\textbf{Value:} 3\quad \textbf{Priority:} Must\\
\textbf{Acceptance Criteria} 
\begin{itemize}
  \item Authenticated users with the developer or designer role should be able to view the dashboard screen.
  \item It should contain a view for each implemented dashboard view. 
  \item Each view should display the information calculated for that view. 
\end{itemize}
\paragraph{Suggestions}
\textbf{Value:} 1\quad \textbf{Priority:} Should\\
\textbf{Acceptance Criteria} 
\begin{itemize}
  \item The system should contain a tab for displaying design suggestions. 
  \item In this tab, rule based suggestions must be displayed to the authenticated user.
\end{itemize}
\paragraph{Decision Log}
\textbf{Value:} 0\quad \textbf{Priority:} Wont\\
\textbf{Acceptance Criteria} N/A
\subsubsection{Authentication and Acess Control}
\paragraph{Login}
\textbf{Value:} 5\quad \textbf{Priority:} Must\\
\textbf{Acceptance Criteria}
\begin{itemize}
  \item Users must be required to log in before accessing the python application.
  \item Users with an account must be able to log into that account by entering the correct details.
  \item Users without an account must be able to create one.
\end{itemize}
\paragraph{Password Reset}
\textbf{Value:} 1\quad \textbf{Priority:} Could\\
\textbf{Acceptance Criteria}
\begin{itemize}
  \item Users must be able to reset their login credentials.
  \item This reset system must require them to have some other security factor such as email.
\end{itemize}
\paragraph{Player Permissions}
\textbf{Value:} 1\quad \textbf{Priority:} Must\\
\textbf{Acceptance Criteria}
\begin{itemize}
  \item Users with the player role must be authenticated as players when logging in.
  \item A player must not be able to view any of the dashboards. 
\end{itemize}
\paragraph{Designer Permissions}
\textbf{Value:} 1\quad \textbf{Priority:} Must\\
\textbf{Acceptance Criteria}
\begin{itemize}
  \item Users with the designer role must be authenticated as designers when logging in.
  \item A designer must be able to view the dashboards.
  \item A designer must be able to view the design suggestions.
\end{itemize}
\paragraph{Developer Permissions}
\textbf{Value:} 1\quad \textbf{Priority:} Must\\
\textbf{Acceptance Criteria}
\begin{itemize}
  \item Users with the developer role must be authenticated as developers when logging in.
  \item A developer must be able to perform any action in the Python application a designer could. 
\end{itemize}
\subsubsection{Dashboard Views}
\paragraph{Simulation View}
\textbf{Value:} 3\quad \textbf{Priority:} Must\\
\textbf{Acceptance Criteria}
\begin{itemize}
  \item Authenticated designers and developers must be able to view the results of the most recently ran simulation in the dashboard views.
\end{itemize}
\paragraph{Funnel View}
\textbf{Value:} 3\quad \textbf{Priority:} Must\\
\textbf{Acceptance Criteria}
\begin{itemize}
  \item The system must calculate a funnel view showing user character drop off as the game progresses from the telemetry data.
  \item This must then be passed to the user interface to display.
\end{itemize}
\paragraph{Difficulty Spikes}
\textbf{Value:} 3\quad \textbf{Priority:} Must\\
\textbf{Acceptance Criteria}
\begin{itemize}
  \item The system must calculate a difficulty spikes view showing failure rate for different stages from the telemetry data.  
  \item This must then be passed to the user interface to display.
\end{itemize}
\paragraph{Progress Curves}
\textbf{Value:} 3 \quad \textbf{Priority:} Could\\
\textbf{Acceptance Criteria} 
\begin{itemize}
  \item The system must calculate how many coins user's character's accumulated on each stage, and the time to complete each stage from the telemetry data.
  \item This must then be passed to the user interface to display.
\end{itemize}
\paragraph{Fairness Indicators}
\textbf{Value:} 0 \quad \textbf{Priority:} Wont\\
\textbf{Acceptance Criteria} N/A
\paragraph{Comparison Mode}
\textbf{Value:} 3 \quad \textbf{Priority:} Could\\
\textbf{Acceptance Criteria} 
\begin{itemize}
  \item The system must calculate how many coins user's character's accumulated on each stage along with how much health they finish each stage with, for each difficulty.
  \item This must then be passed to the user interface to display.
\end{itemize}
\subsubsection{Design Suggestions}
\paragraph{Rule Based Design Suggestions}
\textbf{Value:} 1 \quad \textbf{Priority:} Should\\
\textbf{Acceptance Criteria} 
\begin{itemize}
  \item The system should read the telemetry data and evaluate it against a suggestion criteria.
  \item If this criteria is met, the system should suggest a concrete change to the design parameter to improve the balance of the game. 
\end{itemize}
\subsubsection{Telemetry Events}
\paragraph{Read Telemetry Events}
\textbf{Value:} 5 \quad \textbf{Priority:} Must\\
\textbf{Acceptance Criteria} 
\begin{itemize}
  \item The system must be able to read telemetry events from the telemetry store for actual users.
  \item The system must be able to read telemetry events from the telemetry store for simulated runs.
\end{itemize}
\paragraph{Validate Telemetry Events}
\textbf{Value:} 5 \quad \textbf{Priority:} Must\\
\textbf{Acceptance Criteria} 
\begin{itemize}
  \item The system must be able to detect missing fields in read in telemetry events and ignore them.
  \item The system must be able to detect telemetry events with invalid field values and ignore them.
\end{itemize}
\subsubsection{Seeded Dataset}
\paragraph{Telemetry Events}
\textbf{Value:} 0 \quad \textbf{Priority:} Wont\\
\textbf{Acceptance Criteria} N/A
\paragraph{Stage Variety}
\textbf{Value:} 0 \quad \textbf{Priority:} Wont\\
\textbf{Acceptance Criteria} N/A
\paragraph{Difficulty Variety}
\textbf{Value:} 0 \quad \textbf{Priority:} Wont\\
\textbf{Acceptance Criteria} N/A
\paragraph{User Variety}
\textbf{Value:} 0 \quad \textbf{Priority:} Wont\\
\textbf{Acceptance Criteria} N/A
\paragraph{Session Variety}
\textbf{Value:} 0 \quad \textbf{Priority:} Wont\\
\textbf{Acceptance Criteria} N/A
\paragraph{Anomalous Telemetry Events}
\textbf{Value:} 0 \quad \textbf{Priority:} Wont\\
\textbf{Acceptance Criteria} N/A
\paragraph{Balancing Decisions}
\textbf{Value:} 0 \quad \textbf{Priority:} Wont\\
\textbf{Acceptance Criteria} N/A
\subsection{Java Application}
\subsubsection{User Interface}
\paragraph{Login}
\textbf{Value:} 2 \quad \textbf{Priority:} Must\\
\textbf{Acceptance Criteria}
\begin{itemize}
  \item Users of the java application should first see a login screen.
  \item This screen must prompt the users to sign in before they can access any other screen.
  \item Users should also be able to reset their login or create an account from this screen.
\end{itemize}
\paragraph{Main Menu}
\textbf{Value:} 2 \quad \textbf{Priority:} Must\\
\textbf{Acceptance Criteria}
\begin{itemize}
  \item Authenticated users must be able to view a main menu screen once authenticated.
  \item From this screen they must be able to start a run.
  \item From this screen they must be able to view settings.
  \item From this screen they must be able to close the game. 
\end{itemize}
\paragraph{Settings}
\textbf{Value:} 3 \quad \textbf{Priority:} Must\\
\textbf{Acceptance Criteria}
\begin{itemize}
  \item Authenticated users must be able to view a the settings menu.
  \item It must contain the option to toggle whether a user has telemetry enabled.
  \item It must contain the option to view the telemetry disclosure.
  \item It must contain the option to change implemented design parameter values for designers and developers.
  \item It must contain the option to execute a simulated run for designers and developers.
  \item It must contain the option to assign roles to other users for developers.
\end{itemize}
\paragraph{Start Run}
\textbf{Value:} 1 \quad \textbf{Priority:} Must\\
\textbf{Acceptance Criteria}
\begin{itemize}
  \item Authenticated users must be able to view the start run menu from the main menu.
  \item It must allow users to select the difficulty for their run.
  \item Once difficulty is selected it must place them in an encounter.
\end{itemize}
\paragraph{Encounter}
\textbf{Value:} 2 \quad \textbf{Priority:} Must\\
\textbf{Acceptance Criteria}
\begin{itemize}
  \item Authenticated users must be able to view the encounter menu while in a run.
  \item It must inform them what enemies are in the encounter.
  \item It must inform the play of how much health, magic, coins and lives they have.
  \item It must allow the user to select which attack ability they will use.
  \item It must allow the user to select which enemy they are attacking.
\end{itemize}
\paragraph{Shop}
\textbf{Value:} 2 \quad \textbf{Priority:} Must\\
\textbf{Acceptance Criteria}
\begin{itemize}
  \item Authenticated user must be able to view the shop after completing an encounter, that was not the last encounter.
  \item The shop must inform the user of how many coins they have.
  \item The shop must inform the user of what abilities are available for them to purchase.
  \item The shop must inform the user of the cost of each of these abilities.
  \item The shop must allow the user to select 1 ability to buy.
  \item The shop should allow users to leave without buying an ability.
  \item Once an ability is bought the user should enter the next stage, and it's encounter.
\end{itemize}

\paragraph{End Run}
\textbf{Value:} 2 \quad \textbf{Priority:} Must\\
\textbf{Acceptance Criteria}
\begin{itemize}
  \item Authenticated user must be able to view the end run screen, once a run is over.
  \item It must display the stage at which the run ended.
  \item It must display how long the run took.
  \item It must display how many times the user's character died.
  \item It must display how many coins the user's character had at the end of the run.
  \item It must allow the user to return to the main menu.
\end{itemize}
\subsubsection{Autheentication and Acesss Control}
\paragraph{Login}
\textbf{Value:} 5\quad \textbf{Priority:} Must\\
\textbf{Acceptance Criteria}
\begin{itemize}
  \item Users must be required to log in before accessing the java application.
  \item Users with an account must be able to log into that account by entering the correct details.
  \item Users without an account must be able to create one.
\end{itemize}
\paragraph{Password Reset}
\textbf{Value:} 1\quad \textbf{Priority:} Could\\
\textbf{Acceptance Criteria}
\begin{itemize}
  \item Users must be able to reset their login credentials.
  \item This reset system must require them to have some other security factor such as email.
\end{itemize}
\paragraph{Player Permissions}
\textbf{Value:} 1\quad \textbf{Priority:} Must\\
\textbf{Acceptance Criteria}
\begin{itemize}
  \item Users with the player role must be authenticated as players when logging in.
  \item A player must be able to view settings, only having access to the telemetry disclose and the ability to toggle telemetry.
  \item A player must be able to start a game run, and play the game. 
\end{itemize}
\paragraph{Designer Permissions}
\textbf{Value:} 1\quad \textbf{Priority:} Must\\
\textbf{Acceptance Criteria}
\begin{itemize}
  \item Users with the designer role must be authenticated as designers when logging in.
  \item A designer must be able to do everything a player can.
  \item A designer must be able to view and edit the design parameters in settings.
  \item A designer must be able to run a simulation from settings, which's telemetry can be viewed in the Python application. 
\end{itemize}
\paragraph{Developer Permissions}
\textbf{Value:} 1\quad \textbf{Priority:} Must\\
\textbf{Acceptance Criteria}
\begin{itemize}
  \item Users with the developer role must be authenticated as developers when logging in.
  \item A developer must be able to do everything a designer can do.
  \item A developer must also be able to assign roles to users from settings. 
\end{itemize}
\subsubsection{Settings}
\paragraph{Telemetry}
\textbf{Value:} 3\quad \textbf{Priority:} Must\\
\textbf{Acceptance Criteria}
\begin{itemize}
  \item Users must be able to toggle whether telemetry is enabled.
  \item When telemetry is disabled, no telemetry events relating to the user should be stored. 
  \item Whether they have telemetry enabled or not must persist beyond the runtime of the application.
\end{itemize}
\paragraph{Design Parameters}
\textbf{Value:} 3\quad \textbf{Priority:} Must\\
\textbf{Acceptance Criteria}
\begin{itemize}
  \item Designers must be able to change how many lives a user's character starts a run with. 
  \item This change must be stored and persist outside the runtime of the application.
\end{itemize}
\paragraph{Simulation Mode}
\textbf{Value:} 5\quad \textbf{Priority:} Must\\
\textbf{Acceptance Criteria}
\begin{itemize}
  \item Designers must be able to execute a simulated run of the game.
  \item The simulation must store telemetry events for viewing in the python application.
  \item These events must be kept separate from user generated events.
\end{itemize}
\paragraph{Role Asignment}
\textbf{Value:} 1\quad \textbf{Priority:} Must\\
\textbf{Acceptance Criteria}
\begin{itemize}
  \item Developers must be able assign roles to other users.
  \item They must provide the user's username and what role to update the user to.
  \item The user must then have their role updated to the new role.
\end{itemize}
\subsubsection{Gameplay Loop}
\paragraph{Start Run}
\textbf{Value:} 1\quad \textbf{Priority:} Must\\
\textbf{Acceptance Criteria}
\begin{itemize}
  \item Authenticated users must be able to start a new game run.
  \item The run must have a fresh user character with only the basic abilities defined in the game design documentation.
  \item The run must start on stage one. 
\end{itemize}
\paragraph{Chose Difficulty}
\textbf{Value:} 1\quad \textbf{Priority:} Must\\
\textbf{Acceptance Criteria}
\begin{itemize}
  \item Once a run is started the user must select which difficulty they want the run to be.
  \item This cannot be changed once the run has started.
\end{itemize}
\paragraph{Progress Through Stages}
\textbf{Value:} 1\quad \textbf{Priority:} Must\\
\textbf{Acceptance Criteria}
\begin{itemize}
  \item After difficulty is chosen the user should start in stage one.
  \item After completing each encounter and leaving the shop, the user should progress to the next stage.
  \item Stage one and two must draw their encounter from the same pool.
  \item Once stage two is completed the game ends. 
\end{itemize}
\paragraph{Be in an Encounter}
\textbf{Value:} 1\quad \textbf{Priority:} Must\\
\textbf{Acceptance Criteria}
\begin{itemize}
  \item Once difficulty is selected the user's character must enter an encounter. 
  \item At the start of each stage the user's character must enter an encounter.
  \item An encounter must consist of one or more enemies.
  \item There must be two encounters in the game
\end{itemize}
\paragraph{Use Physcal Attack Abilities}
\textbf{Value:} 2\quad \textbf{Priority:} Must\\
\textbf{Acceptance Criteria}
\begin{itemize}
  \item While in an encounter, on the user's character's turn, they must be able to use physical attack abilities.
  \item These must cost no magic to use.
  \item They must target one enemy which the user choses.
  \item They must deal damage to the enemy, of a damage type specified by the ability.
\end{itemize}
\paragraph{Use Magical Attack Abilities}
\textbf{Value:} 2\quad \textbf{Priority:} Must\\
\textbf{Acceptance Criteria}
\begin{itemize}
  \item While in an encounter, on the user's character's turn, they must be able to use magical attack abilities.
  \item These must cost some amount of magic to use, as specified in the game design documentation.
  \item They must target one enemy which the user choses.
  \item They must deal damage to the enemy, of a damage type specified by the ability.
  \item They must not be usable if the user's character does not have enough magic. 
\end{itemize}
\paragraph{Take Turns}
\textbf{Value:} 1\quad \textbf{Priority:} Must\\
\textbf{Acceptance Criteria}
\begin{itemize}
  \item While in an encounter, the user should take their turn first.
  \item Once the user has used an attack ability that is the end of their turn.
  \item After the user's turn, each enemy in the encounter should take their turn sequentially.
  \item Each enemy should chose one attack ability and use it against the player, on the enemy's turn.
  \item Once all the enemies have taken their turn, it should be the player's turn again.
  \item This loop should continue until the encounter is complete.
\end{itemize}
\paragraph{Kill Enemies}
\textbf{Value:} 2\quad \textbf{Priority:} Must\\
\textbf{Acceptance Criteria}
\begin{itemize}
  \item Each enemy within an encounter must have a finite amount of health.
  \item Each time an enemy takes damage their health must be reduced by an amount based on the damage dealt, and it's damage type, as specified in the game design documentation.
  \item When an enemy's health is less than or equal to 0 they must die. 
  \item A dead enemy can no longer nave turns.
\end{itemize}
\paragraph{Complete an Encounter}
\textbf{Value:} 1\quad \textbf{Priority:} Must\\
\textbf{Acceptance Criteria}
\begin{itemize}
  \item While in an encounter, if all enemies are dead the encounter should be marked complete.
  \item A user should never get an encounter in a run, if they've already completed that encounter.
  \item Once an encounter is complete the user must move on to the shop.
  \item Upon completing an encounter the user should gain coins as specified in the game design documentation.
\end{itemize}
\paragraph{View Shop}
\textbf{Value:} 1\quad \textbf{Priority:} Must\\
\textbf{Acceptance Criteria}
\begin{itemize}
  \item While in the shop the system must inform the user interface what items are available to purchase.
  \item The system must also provide their price. 
\end{itemize}
\paragraph{Purchase Upgrades in the Shop}
\textbf{Value:} 2\quad \textbf{Priority:} Must\\
\textbf{Acceptance Criteria}
\begin{itemize}
  \item While in the shop the user must be able to purchase ability if they have enough coins to afford them.
  \item A purchased ability should then be applied to the user's character, either granting them a passive or active ability.
\end{itemize}
\paragraph{Complete a Run}
\textbf{Value:} 2\quad \textbf{Priority:} Must\\
\textbf{Acceptance Criteria}
\begin{itemize}
  \item Once the user completes the encounter in the second stage, the run is complete.
  \item If the user's character runs out of lives the run should also be completed, although as a failure rather than success. 
  \item The system must have tracked the statistics related to the run that are to be displayed to the user at the run end screen.
  \item The system must then provide these statistics to display on the user interface.   
\end{itemize}
\subsubsection{Game Mechanics}
\paragraph{Coins}
\textbf{Value:} 2\quad \textbf{Priority:} Must\\
\textbf{Acceptance Criteria}
\begin{itemize}
  \item At the end of each encounter the user's character must obtain a number of coins specified in the game design documentation.
  \item While in the shop the user can send these coins to obtain abilities.
  \item Once an ability is bought it is applied to the user's character.
\end{itemize}
\paragraph{Health}
\textbf{Value:} 2\quad \textbf{Priority:} Must\\
\textbf{Acceptance Criteria}
\begin{itemize}
  \item All entities within the game must have a finite amount of health.
  \item When an entity takes damage their health must be reduced by an amount based on the damage and damage type, as specified in the game design documentation.
  \item When an enemy's health is less than or equal to 0, they die.
  \item When the user's character's health is less than or equal to 0, they lose a life. 
\end{itemize}
\paragraph{Lives}
\textbf{Value:} 2\quad \textbf{Priority:} Must\\
\textbf{Acceptance Criteria}
\begin{itemize}
  \item The user's character has a finite number of lives.
  \item Each time the user's character dies they lose a life.
  \item Once the user's character runs out of the lives the run ends. 
\end{itemize}
\paragraph{Damage Types}
\textbf{Value:} 3\quad \textbf{Priority:} Must\\
\textbf{Acceptance Criteria}
\begin{itemize}
  \item There must be physical damage, which is caused by any physical attack such as a slash or punch.
  \item There must be fire damage, which is caused by any burning attack such as a fire ball.
  \item There must be water damage, which is caused by any water attack such as water jet.
  \item There must be thunder damage, which is caused by any stormy attack such as thunder storm.
  \item There must be absolute damage, which is caused by any source of damage that cannot be resisted, such as absolute pulse.
  \item For all damage types, there must be a way to modify how much of that damage type entities deal from attack abilities.
  \item For all damage types, there must be a way to modify how much of that damage type entities take when targeted by attack abilities. 
  \item Absolute damage can not be reduced in any way. 
\end{itemize}
\paragraph{Magic}
\textbf{Value:} 3\quad \textbf{Priority:} Must\\
\textbf{Acceptance Criteria}
\begin{itemize}
  \item The player must have a finite amount of magic.
  \item The player must have a maximum amount of magic they can hold.
  \item At the start of each of the player's turns they must gain an amount of magic specified by the game design documentation. 
  \item The user must not be able to use a magic attack ability if they do not have enough magic to fuel it.
  \item The user must be able to use a magic attack ability if they have enough magic to fuel it. Once used it must reduce their magic by it's cost. 
\end{itemize}
\paragraph{Passive Abilities}
\textbf{Value:} 3\quad \textbf{Priority:} Must\\
\textbf{Acceptance Criteria}
\begin{itemize}
  \item The user must be able to buy passive passive abilities in the shop.
  \item Once bought the user's character must gain the benefits of the passive ability. 
  \item There must be a passive ability for each damage type, that allows the user's character to resist that damage type, in accordance with the game design documentation. 
\end{itemize}
\paragraph{Attack Abilities}
\textbf{Value:} 3\quad \textbf{Priority:} Must\\
\textbf{Acceptance Criteria}
\begin{itemize}
  \item The user must be able to buy attack passive abilities in the shop.
  \item Once bought the user's character must gain the ability to use the attack ability in an encounter. 
  \item The prototype must feature all attack abilities outlined in the game design documentation. 
\end{itemize}
\paragraph{Difficulties}
\textbf{Value:} 3\quad \textbf{Priority:} Must\\
\textbf{Acceptance Criteria}
\begin{itemize}
  \item The game must feature three difficulties: easy, moderate, and hard. 
  \item Easy difficulty must be the easiest.
  \item Moderate difficulty must be harder than easy, but easier than hard.
  \item Hard difficulty must be the hardest. 
  \item Each difficulty must allow different values for design parameters, to allow it to adjust the difficulty of the game.  
\end{itemize}
\subsubsection{Telemetry Events}
\paragraph{Event Writing}
\textbf{Value:} 3\quad \textbf{Priority:} Must\\
\textbf{Acceptance Criteria}
\begin{itemize}
  \item The game must validate if telemetry is enabled before writing telemetry events. If it is not no events should be written. If it is the game should write telemetry events.
  \item The game should write each telemetry event to a shared telemetry store that the python application can read from.
  \item The game should only lock the file while writing telemetry events. 
\end{itemize}
\paragraph{Event Validation}
\textbf{Value:} 3\quad \textbf{Priority:} Could\\
\textbf{Acceptance Criteria}
\begin{itemize}
  \item The game could verify if incoming telemetry events are missing fields and not write them if they do.
  \item The game could verify incoming telemetry events do not have timestamps in an impossible order (e.g. end session before start session)
  \item The game could verify events with the same session ID all come from the same session, start with a start session event and end with an end session event, and discard them if they do not.
  \item The game could verify that the user ID field of all telemetry events matches the user ID of the currently authenticated user, and if it does not, discard them.
  \item The game could verify that outgoing telemetry events do not come from a timestamp in the future, and discard them if they do. 
\end{itemize}
\paragraph{Start Session}
\textbf{Value:} 1\quad \textbf{Priority:} Must\\
\textbf{Acceptance Criteria}
\begin{itemize}
  \item The game must feature the start session event.
  \item The event must have the fields specified in the telemetry schema.
  \item The event must be created when a new run starts.
\end{itemize}
\paragraph{Normal Encounter Start}
\textbf{Value:} 1\quad \textbf{Priority:} Must\\
\textbf{Acceptance Criteria}
\begin{itemize}
  \item The game must feature the normal encounter start event.
  \item The event must have the fields specified in the telemetry schema.
  \item The event must be created each time a normal encounter starts, including restarts.
\end{itemize}
\paragraph{Normal Encounter Fail}
\textbf{Value:} 1\quad \textbf{Priority:} Must\\
\textbf{Acceptance Criteria}
\begin{itemize}
  \item The game must feature the normal encounter fail event.
  \item The event must have the fields specified in the telemetry schema.
  \item The event must be created each time a normal encounter is failed by the user's character dying.
\end{itemize}
\paragraph{Normal Encounter Complete}
\textbf{Value:} 1\quad \textbf{Priority:} Must\\
\textbf{Acceptance Criteria}
\begin{itemize}
  \item The game must feature the normal encounter complete event.
  \item The event must have the fields specified in the telemetry schema.
  \item The event must be created each time the player completes a normal encounter by killing all the enemies.
\end{itemize}
\paragraph{Boss Encounter Start}
\textbf{Value:} 0\quad \textbf{Priority:} Wont\\
\textbf{Acceptance Criteria} N/A
\paragraph{Boss Encounter Fail}
\textbf{Value:} 0\quad \textbf{Priority:} Wont\\
\textbf{Acceptance Criteria} N/A
\paragraph{Boss Encounter Complete}
\textbf{Value:} 0\quad \textbf{Priority:} Wont\\
\textbf{Acceptance Criteria} N/A
\paragraph{Gain Coin}
\textbf{Value:} 0\quad \textbf{Priority:} Wont\\
\textbf{Acceptance Criteria} N/A
\paragraph{Buy Upgrade}
\textbf{Value:} 0\quad \textbf{Priority:} Wont\\
\textbf{Acceptance Criteria} N/A
\paragraph{End Session}
\textbf{Value:} 1\quad \textbf{Priority:} Must\\
\textbf{Acceptance Criteria} 
\begin{itemize}
  \item The game must feature the End session event.
  \item The event must have the fields specified in the telemetry schema.
  \item The event must be created when a game run ends, either by the user completing the run by completing the second encounter, or by their character dying.
\end{itemize}
\paragraph{Settings Change}
\textbf{Value:} 1\quad \textbf{Priority:} Must\\
\textbf{Acceptance Criteria} 
\begin{itemize}
  \item The game must feature the settings change event.
  \item The event must have the fields specified in the telemetry schema.
  \item The event must be created when the user changes a setting's value and saves this change. 
\end{itemize}
\paragraph{Kill Enemy}
\textbf{Value:} 0\quad \textbf{Priority:} Wont\\
\textbf{Acceptance Criteria} N/A